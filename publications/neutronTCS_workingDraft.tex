\documentclass[twocolumn,secnumarabic,amssymb, nobibnotes, aps, prl,
superscriptaddress, nobalancelastpage]{revtex4}
\newcommand{\revtex}{REV\TeX\ }
\newcommand{\classoption}[1]{\texttt{#1}}
\newcommand{\macro}[1]{\texttt{\textbackslash#1}}
\newcommand{\m}[1]{\macro{#1}} \newcommand{\env}[1]{\texttt{#1}}
\setlength{\textheight}{9.5in} \usepackage{graphicx}
\setlength{\belowcaptionskip}{6pt}
\usepackage{amsmath} \usepackage{braket} \usepackage{epsfig}
\usepackage{upgreek}
\usepackage{textcomp}
\usepackage[free-standing-units]{siunitx}

\newcolumntype{N}{>{\centering\arraybackslash}m{1.3in}}
\newcolumntype{M}{>{\centering\arraybackslash}m{1.0in}}
\newcolumntype{G}{>{\centering\arraybackslash}m{0.5in}}
\newcolumntype{R}{>{\raggedleft\arraybackslash}m{0.4in}}

\newcommand{\tot}{\ensuremath{\sigma_{tot}}}
\newcommand{\totRD}{\ensuremath{\sigma_{A,A'}}(E)}
\newcommand{\react}{\ensuremath{\sigma_{react}}}
\newcommand{\elast}{\ensuremath{\sigma_{el}}}

\newcommand{\oSix}{\ensuremath{^{16}}O}
\newcommand{\oSeven}{\ensuremath{^{17}}O}
\newcommand{\oEight}{\ensuremath{^{18}}O}
\newcommand{\oSixEight}{\ensuremath{^{16,18}}O}

\newcommand{\neEight}{\ensuremath{^{18}}N\lowercase{e}}

\newcommand{\caForty}{\ensuremath{^{40}}C\lowercase{a}}
\newcommand{\caEight}{\ensuremath{^{48}}C\lowercase{a}}
\newcommand{\caAughtEight}{\ensuremath{^{40,48}}C\lowercase{a}}

\newcommand{\niSix}{\ensuremath{^{56}}N\lowercase{i}}
\newcommand{\niEight}{\ensuremath{^{58}}N\lowercase{i}}
\newcommand{\niSixty}{\ensuremath{^{60}}N\lowercase{i}}
\newcommand{\niOne}{\ensuremath{^{61}}N\lowercase{i}}
\newcommand{\niTwo}{\ensuremath{^{62}}N\lowercase{i}}
\newcommand{\niFour}{\ensuremath{^{64}}N\lowercase{i}}
\newcommand{\niEightFour}{\ensuremath{^{58,64}}N\lowercase{i}}
\newcommand{\niNat}{\ensuremath{^{\text{nat}}}N\lowercase{i}}

\newcommand{\rhThree}{\ensuremath{^{103}}R\lowercase{h}}

\newcommand{\snTwelve}{\ensuremath{^{112}}S\lowercase{n}}
\newcommand{\snFourteen}{\ensuremath{^{114}}S\lowercase{n}}
\newcommand{\snFifteen}{\ensuremath{^{115}}S\lowercase{n}}
\newcommand{\snSixteen}{\ensuremath{^{116}}S\lowercase{n}}
\newcommand{\snSeventeen}{\ensuremath{^{117}}S\lowercase{n}}
\newcommand{\snEighteen}{\ensuremath{^{118}}S\lowercase{n}}
\newcommand{\snNineteen}{\ensuremath{^{119}}S\lowercase{n}}
\newcommand{\snTwenty}{\ensuremath{^{120}}S\lowercase{n}}
\newcommand{\snTwo}{\ensuremath{^{122}}S\lowercase{n}}
\newcommand{\snFour}{\ensuremath{^{124}}S\lowercase{n}}
\newcommand{\snTwelveFour}{\ensuremath{^{112,124}}S\lowercase{n}}
\newcommand{\snNat}{\ensuremath{^{\text{\lowercase{nat}}}}S\lowercase{n}}
\newcommand{\snTwelveNatFour}{\ensuremath{^{112,\text{nat},124}}S\lowercase{n}}

\newcommand{\pbEight}{\ensuremath{^{208}}P\lowercase{b}}
\newcommand{\pbNat}{\ensuremath{^{\text{\lowercase{nat}}}}P\lowercase{b}}

% shell labels
\newcommand{\sOne}{s\ensuremath{_{\frac{1}{2}}}}
\newcommand{\pThree}{p\ensuremath{_{\frac{3}{2}}}}
\newcommand{\pOne}{p\ensuremath{_{\frac{1}{2}}}}
\newcommand{\dFive}{d\ensuremath{_{\frac{5}{2}}}}
\newcommand{\dThree}{d\ensuremath{_{\frac{3}{2}}}}
\newcommand{\fSeven}{f\ensuremath{_{\frac{7}{2}}}}
\newcommand{\fFive}{f\ensuremath{_{\frac{5}{2}}}}
\newcommand{\gNine}{g\ensuremath{_{\frac{9}{2}}}}
\newcommand{\gSeven}{g\ensuremath{_{\frac{7}{2}}}}
\newcommand{\hEleven}{h\ensuremath{_{\frac{11}{2}}}}

\newcommand{\tZero}{T\ensuremath{_{0}}}



\begin{document}

\begin{abstract}
    The neutron total cross sections \tot\ of $^{16,18}$O,
    $^{58,nat,64}$Ni, $^{103}$Rh, and $^{112,nat,124}$Sn have been measured at the Los Alamos
    Neutron Science Center (LANSCE) at intermediate energies (3 $\leq E_{n}
    \leq$ 450 MeV) by
    leveraging waveform digitizer technology. The results are in good agreement
    with previous measurements that used analog techniques,
    excepting small deviations at high energies. These data
    continue the campaign of
    \tot\ measurements we initiated with the case study of $^{40,48}$Ca in 2009.
    The \tot\ relative differences between isotopes are presented,
    revealing additional information about
    the isovector components needed for an accurate optical model (OM)
    description away from stability. Digitizer-enabled \tot-measurement
    techniques are discussed and a Dispersive Optical Model (DOM)
    analysis using these data to extract neutron skins and spectroscopic factors
    is presented.
\end{abstract}

\title{Isotopically-Resolved Neutron Total Cross Sections At
Intermediate Energies}

\author{C.~D.~Pruitt}  \email[Corresponding author:]{cdpruitt@wustl.edu}
\author{R.~J.~Charity}
\author{D. E.~M.~Hoff}  
\author{L.~G.~Sobotka}
\author{K.~W.~Brown} \altaffiliation{Present Address: \textit{National
        Superconducting Cyclotron Laboratory, Departments of Physics and
Astronomy, Michigan State University, East Lansing, MI 48824, USA}}
\author{J.~M.~Elson}
\affiliation{Department of Chemistry, Washington University, St. Louis, MO 63130}

\author{H. Y. Lee}
\author{M. Devlin}
\author{N. Fotiadis}
\author{S. Mosby}
\affiliation{Los Alamos National Lab, Los Alamos, NM 87545, USA}
\maketitle

Neutron scattering is a direct, Coulomb-insensitive tool for probing the nuclear
environment. The simplest neutron-nucleus interaction quantity is 
the neutron total cross section, \tot, which provides fundamental information about
nuclear size and the ratio of elastic-to-inelastic components of nucleon 
scattering. Additionally, \tot\ data are thought to be tightly correlated with
a variety of structural nuclear properties of great interest
including the neutron skin of neutron-rich nuclei
\cite{Mahzoon2017} and thus the density dependence of the symmetry energy $L$,
an essential equation-of-state (EOS) input for neutron star
structure calculations \cite{Fattoyev2012, Vinas2014, Brown2000}.

In the crudest approximation, the ``strongly-absorbing sphere'' picture
explains the dependence of \tot\ on the size of the target nucleus and on the
energy of the incident neutron as a consequence of simple size-scaling:
\begin{equation} \label{SASAbsolute}
    \sigma_{tot}(E) = 2\pi(R + \lambdabar)^{2},
\end{equation}
where $R=r_{0}A^{\frac{1}{3}}$ and $\lambdabar$ is the reduced wavelength
of the incident neutron with energy $E$ in the center of mass \cite{Fernbach1949, Satchler1980}. 
%To compare \tot\ for two different targets of masses A and A$'$, the relative
%difference \totRD\ is useful:
%\begin{equation} \label{SASRelDiff}
%    \begin{split}
%        \totRD\ & \equiv
%    \frac{\sigma_{A}-\sigma_{A'}}{\sigma_{A}+\sigma_{A'}} \\
%    & =
%    \frac{r_{0}^{2}(A^{\frac{2}{3}}-A'^{\frac{2}{3}}) +
%    2\lambdabar r_{0}(A^{\frac{1}{3}}-A'^{\frac{1}{3}})}
%    {r_{0}^{2}(A^{\frac{2}{3}}+A'^{\frac{2}{3}}) +
%    2\lambdabar r_{0}(A^{\frac{1}{3}}+A'^{\frac{1}{3}}) + 2\lambdabar^{2}},
%    \end{split}
%\end{equation}
While on \textit{average}, experimental \tot\ data comport with this naive
model, the most prominent feature of experimental \tot\ data is the oscillatory
behavior centered about the average of Eq. \ref{SASAbsolute}, visible in Fig.
\ref{SASphereVsExperiment}.

\begin{figure}
    \includegraphics[width=0.5\textwidth]{../analysis/plots/theory/SASphereVsExperiment.png}
    \caption{
        (Color online) Experimental \tot\ data are shown from 2-500
        MeV for nuclides from A=12 to A=208
        \cite{Finlay1993, Schwartz1974, Poenitz1983, Abfalterer2000, Abfalterer2001}.
        Predictions for \tot\ given by the ``strongly-absorbing sphere'' (SAS)
        model, Eq. \ref{SASAbsolute}, are shown as thin dashed lines for each nucleus.
        Regular oscillations about the SAS model are clearly visible,
        as is the trend for the oscillation
        maxima and minima to shift to \textit{higher} energies as A is increased.
    }
    \label{SASphereVsExperiment}
\end{figure}
Peterson \cite{Peterson1962} interpreted these oscillations as the 
result of a phase shift between neutron partial waves passing \textit{around} the 
nucleus (thus undergoing no phase shift) and waves passing
\textit{through} the nuclear potential, where they are refracted and exhibit a 
retardation of phase (an illustration is available in \cite{Satchler1980}).
This explanation was termed the 
``nuclear Ramsauer effect" by Carpenter and Wilson \cite{Carpenter1959} based on 
the analogous effect seen in electron scattering on noble gases.

Following Angeli and Csikai \cite{Angeli1970}, this explanation can be
incorporated into the strongly-absorbing sphere relations (Eq. \ref{SASAbsolute}) 
by addition of a sinusoidal term:
\begin{equation} \label{OscillatoryModel}
    \tot = 2\pi (R+\lambdabar)^{2}(1 - \rho \cos(\delta))
\end{equation}
where $\rho = e^{-\operatorname{Im}(\Delta)}$ and $\delta =
\operatorname{Re}(\Delta)$, $\Delta$ being the phase difference between a
partial wave traveling
around and traveling through the nucleus. The large amplitude of the
oscillations ($\rho$) suggests that elastic scattering accounts for a
significant fraction of the total cross section, in turn implying a 
larger mean free path for neutrons through the nucleus 
than might otherwise be expected in the absence of Pauli blocking
\cite{Mohr1955, Feshbach1958}.
Continuing, if we approximate the nucleus with a
real spherical potential of radius $R$ and depth $U$, the total phase shift $\delta$ is:
\begin{equation} \label{phaseShift}
    \delta =
    \frac{\overline{C}\left(\left[{\frac{E+U}{E}}\right]^{\frac{1}{2}}-1\right)}{\lambdabar}
\end{equation}
where $\overline{C} = \frac{4}{3}R$ is the average chord length through the
sphere \cite{Angeli1970}. Rearranging Eq. \ref{phaseShift} in terms of A and E and
discarding leading constants yields:
\begin{equation}
    \delta \propto A^{\frac{1}{3}}\times\left(\sqrt{E+U}-\sqrt{E}\right)
\end{equation}
This form reveals an important relation: as A is increased, to maintain constant 
phase $\delta$, E must also increase \cite{Satchler1980, Peterson1962}. 
This is contrary to a typical resonance condition where an integer number of wavelengths
are fit inside a potential; in that case, to maintain constant phase as A is increased,
E must be decreased. Thus these \tot\ oscillations have been referred to as
``anti-resonances" or ``echoes" \cite{Satchler1980, McVoy1967}.
Other authors \cite{Ahmad1973} have
exposed weaknesses in Angeli and Csikai's interpretation of
Eq. \ref{OscillatoryModel} and have provided a more general semi-empirical
equation for \tot. However, Eq. \ref{OscillatoryModel} is a valuable starting
point for connecting \tot\ with the depth and shape of the nuclear
potential as experienced by neutrons.

By including additional surface, spin-orbit, and other terms, optical models (OMs) have been 
used to successfully reproduce the general features of all manner of single-nucleon scattering 
data across the chart of nuclides up to several hundred MeV \cite{Perey1976,
CH89, KoningDelaroche}. However, despite the excellent agreement with experiment, optical models
involve the interaction of many partial waves with many sometimes-opaque terms
in the potential, complicating intuitive understanding of the underlying
physics at play. In particular, the isovector components of optical potentials
are quite difficult to constrain as they depend on both proton and neutron 
scattering data, one or both of which are often unavailable.

With these considerations in mind, our present goal is twofold: to
provide new isotopically-resolved \tot\ data useful for identifying the 
dependence of optical 
potential terms on nuclear asymmetry, and, using a Dispersive Optical Model
(DOM) analysis of these new \tot\ data along with a large corpus of scattering
and bound-state data, to extract important quantities (neutron skin
thicknesses, spectroscopic factors) for several cornerstone,
closed proton shell nuclei.

%from  Anderson and Grimes were the first to define an isospin-isospin cross section
%$\sigma_{is}$:
%\begin{equation}
%    \sigma = \frac{\sigma_{tot}^{>} - \sigma_{tot}^{<}}{2}
%\end{equation}
%where > and < refer to the isospin parallel ($T_0 + \frac{1}{2}$) and
%anti-parallel ($T_0 - \frac{1}{2}$) projections, with the explicit intention of
%examining the differences in \tot\ between isotopes to learn more about the
%isovector components of the optical potentials \cite{Anderson1990}.

\section{Experimental Considerations}
By scattering secondary radioactive beams off of hydrogen targets in inverse
kinematics, proton-scattering experiments are possible even on highly unstable
nuclides. Because neutrons themselves must be generated as a
secondary radioactive beam, neutron-scattering experiments are restricted to
normal kinematics and \tot\ measurements are possible only for relatively stable
nuclides that can be formed into a target. At present, \tot\ measurements above
the resonance region on nuclides with short half-lives (shorter than the timescale of
days) are technically infeasible for this reason, though a handful have been carried out on
samples with half-lives in the tens to thousands of years \cite{Poenitz1983,
Phillips1980, Foster1971}.

Traditionally, \tot\ measurements have relied on analog techniques for recording
events, techniques that suffer from a large per-event deadtime of
up to several $\micro\second$. Thus for a typical intermediate-energy \tot\ measurement
with dozens or hundreds of energy bins, achieving statistical uncertainty at the
level of 1\% requires a thick sample to attenuate a sizable fraction of the
incident neutron flux. For cross sections in the 1-10 barn range, this means
sample masses of tens of grams \cite{Finlay1993, Abfalterer2001}.
Producing an isotopically-enriched sample of this size is often
prohibitively expensive; indeed, a literature search for isotopically-resolved
\tot\ measurements reveals a paucity of data from 1-300 MeV, even for
closed-shell isotopes of special importance like $^{3,4}$He, $^{64}$Ni, and
$^{204}$Pb (see Table \ref{IsotopicCrossSectionTable}).

\begin{table}[tb]
    \caption[Selected results from a literature study of
    isotopically-resolved \tot\ data using the EXFOR database \cite{EXFORDatabase}]
    {
        Selected results from a literature search for isotopically-resolved
        \tot\ data using the EXFOR database \cite{EXFORDatabase}.
        For the heaviest and lightest stable nuclides in each closed shell in Z, all
        datasets falling at least partially within
        1-500 \mega\electronvolt\ are shown. For elements
        whose natural abundance is $>$90\% of a single isotope (e.g.,
        96.9\% of $^{\text{nat}}$Ca is \caForty), \tot\ data on the natural
        sample was included as ``isotopic''.
    }
    \label{IsotopicCrossSectionTable}
    \centering
    \begin{tabular}{G M R@{\ --\ }l G}
        \hline
        Isotope & \multicolumn{1}{c}{Nat. Abund. [\%]} & \multicolumn{2}{c}{Energies
    [\mega\electronvolt]} & Reference\\

        \hline
        $^{3}$He & \multicolumn{1}{c}{2$\times$10$^{-4}$} & 1.5 & 40 & \cite{Haesner1983}\\
        $^{4}$He & \multicolumn{1}{c}{$>$99.9} & 0.7 & 30 & \cite{Goulding1973}\\
                 &                  & 2   & 40 & \cite{Haesner1983}\\
                 &                  & 77  & 151 & \cite{Measday1966}\\

        \oSix & 99.8                & 0.2 & 49 & \cite{Perey1972}\\
              &                     & 5   & 600 & \cite{Finlay1993}\\

        \oEight & 0.20 & 0.1 & 2.5 & \cite{Vaughn1965}\\
                & & 2.5 & 19 & \cite{Salisbury1965}\\

        \caForty & 96.9 & $<$0.1 & 6.4 & \cite{Johnson1973}\\
                  & & 5.3 & 560 & \cite{Abfalterer2001}\\

        \caEight & 0.187 & 0.6 & 5.2 & \cite{Harvey1985}\\
                  & & 12 & 276 & \cite{Shane2010}\\

        \niEight & 68.1 & $<$0.1 & 68 & \cite{Perey1993}\\

        \niFour & 0.926 & \multicolumn{2}{c}{14.1} & \cite{Dukarevich1967}\\

        \snTwelve & 0.97 & $<$0.1 & 1.4 & \cite{Timokhov1989}\\
                  & & \multicolumn{2}{c}{14.1} & \cite{Dukarevich1967}\\

        \snFour & 5.79 & 0.3 & 5.0 & \cite{Harper1982}\\
                   & & 5.1 & 26 & \cite{Rapaport1980}\\

        $^{204}$Pb & 1.4 & $<$0.1 & 27 & \cite{Carlton2003}\\

        \pbEight & 52.4 & $<$0.1 & 695 & \cite{Harvey1999}\\
                   & & 5 & 600 & \cite{Finlay1993}\\

        \hline
    \end{tabular}
\end{table}

Recent developments in waveform digitizer technology have made it
possible to reduce the per-event deadtime by an order of magnitude or more,
enabling a corresponding reduction in the necessary sample size. In 2008, with
these improvements available, we
embarked on a campaign of \tot\ measurements on isotopically-enriched samples,
starting with $^{40,48}$Ca from $15 \leq E_{n} \leq 300$ MeV \cite{Shane2010}.
The data from that measurement have been incorporated into a comprehensive
Dispersive Optical Model (DOM) analysis \cite{Mueller2011, Mahzoon2014,
MahzoonPhDThesis} yielding proton and neutron spectroscopic factors, charge
radii, and initial estimates of the neutron skins \cite{Mahzoon2017}
for these nuclei.
Here we advance that program with \tot\ results for
the important closed-shell nuclides
$^{16,18}$O, $^{58,64}$Ni, and $^{112,124}$Sn. We also present a measurement
on a very thin sample of the naturally-monoisotopic $^{103}$Rh to demonstrate that
\tot\ experiments over a broad energy range using minute amounts of material are feasible.

\section{Experimental Details}
All \tot\ measurements were carried out at the 15R
beamline at the Weapons Neutron Research (WNR) facility of the Los Alamos
Neutron Science Center (LANSCE) during two run cycles (November 2016 and
September 2017). Our experiment was modeled on previous
\tot\ measurements at WNR \cite{Finlay1993,Abfalterer2001,Shane2010}.
At WNR,
broad-spectrum neutrons up
to $\approx$700 MeV are generated by impinging proton pulses onto a water-cooled, 7.5
cm-long tungsten target (see Fig. \ref{ExperimentalApparatus}). Before the beam
enters the experimental area, a
permanent magnet deflects all charged particles generated by the proton pulses, 
allowing only neutrons and $\gamma$ rays to reach the flight path. At the
entrance to the flight path, the beam was collimated to 0.200 inches using steel
donuts with a total thickness of 24 inches and hardened using a plug of Hevimet (90\% W, 6\% 
Ni, 4\% Cu by weight) inserted at the upstream entrance of the
collimation stack. After collimation, the beam passed successively through a flux 
monitor, the sample of interest held in a sample changer, a veto detector, and finally the 
time-of-flight (TOF) detector approximately 25 meters from the neutron source.
All detectors consisted of BC-400 fast scintillating plastic mated with 
photomultiplier tubes (PMTs) and encased in either a plastic or
an aluminum housing. The flux monitor and veto detector each had
scintillator thicknesses of $\frac{1}{4}$ inch and the TOF detector had a
scintillator thickness of 1 inch. Signals from all detectors and
the target changer were relayed to a 500-MHz CAEN DT-5730 waveform digitizer
running custom software. To improve time resolution, the TOF detector used two
PMTs (one left, one right) mated to the plastic scintillator and the PMTs' signals were 
summed before digitization.

\begin{figure}
    \includegraphics[width=0.3\textwidth]{figures/ExperimentalSetup.png}
    \caption{(Color online) Experimental configuration at WNR facility.
        Samples are cycled into and out of the beam
        using a linear actuator with a period of 150 seconds. Times-of-flight (TOFs) are
    determined by the TOF detector and used to calculate neutron energy.}
    \label{ExperimentalApparatus}
\end{figure}

\begin{figure}
    \includegraphics[width=0.5\textwidth]{figures/beamStructure.png}
    \caption{(Color online) Neutron beam structure at WNR facility.
        ``Macropulses" of protons (row d) are delivered to
        WNR's tungsten Target 4, where they generate neutrons by spallation.
        Each macropulse consists of
        $\approx$350 proton ``micropulses" (row c). Neutrons
        from each micropulse (row b) disperse in
        time as they travel along the flight path so that $\gamma$ rays and high-energy 
    neutrons catch up to low-energy ones from the previous pulse (row a).}
    \label{BeamStructure}
\end{figure}

The particular neutron beam structure at WNR dictates the energy range
achievable for \tot\ measurements (see Fig. \ref{BeamStructure}).
Proton pulse trains, called ``macropulses", are delivered to the tungsten target at 120 Hz.
Each macropulse consists of ~350 individual proton pulses, called ``micropulses", spaced 1.8 
$\mu$s apart. Each micropulse consists of a single proton packet $<$1 ns wide when it 
arrives at the tungsten target that generates $\gamma$ rays and neutrons within a tight
temporal-spatial range. As neutrons from this micropulse travel along the beam path, 
high energy neutrons separate in time from lower-energy neutrons so that neutron
energy can be determined by standard TOF techniques (see \cite{Moore1980} for details).
Because the $\gamma$ rays and high-energy neutrons from later micropulses can
overtake slower neutrons from an earlier micropulse, the distance of the TOF
detector from the neutron source determines both the minimum neutron energy that can be 
unambiguously resolved and the maximum instantaneous neutron flux, critical to correcting
for per-event deadtime.

A programmable sample changer with six positions
was used to cycle each sample into the beam at a regular interval of 150 seconds 
per sample. Once per macropulse, an analog signal from the sample changer
was recorded to indicate its current position.
The flux monitor was used to correct for variations in beam flux between 
macropulses. The veto detector suppressed events from charged-particle production 
in the samples and in air along the flight path.

Custom digitizer software was used to run the 
digitizer in two complementary modes, referred to as ``DPP mode" and ``Waveform 
mode". In DPP mode, triggers were initiated by the digitizer's onboard
peak-sensing firmware. For each trigger, several quantities were recorded: the trigger 
timestamp, two charge integrals over the detected peak with different
integration ranges (32 ns for the short integral, 100 ns for the long integral),
and a 96-ns portion of the raw digitized waveform, referred to as a ``wavelet".
DPP mode was used for the vast majority of the 
experiment and accounts for $\approx$99\% of the total data volume. In waveform mode, 
the digitizer performs no peak-sensing and was externally triggered. Upon 
triggering, the trigger timestamp and a very long wavelet (60 $\micro\second$) 
were recorded. While waveform mode data accounts for only $\approx$1\% of the total data, 
the instantaneous data rate is much higher than in DPP 
mode because hundreds of $\micro\second$ of consecutive waveform samples are 
stored. Roughly once every three seconds, the digitizer was switched to 
waveform mode for one macropulse, then switched back to DPP mode as quickly as
possible (10-40 ms, depending on run configuration).  

Except for the oxygen and rhodium samples, all samples were prepared as right
cylinders 8.25 mm in diameter and ranging from 10-27 mm in length (see
Table \ref{SampleTable} for sample characteristics and Fig. \ref{SamplesImage}
for sample images). For each element under study, a natural-abundance sample
was also prepared as were two natural carbon
samples and a natural lead sample, useful for benchmarking against
literature data. The samples
were inserted into styrofoam sleeves and seated in the cradles of the sample
changer. This design minimizes the amount of non-target mass proximate to the
neutron beam path.

\begin{table*}[tb]
    \caption[Physical characteristics of samples used for neutron \tot\
    measurements]
    {
        For isotopically-enriched samples, the natural abundance
        of the enriched isotope and the isotopic fraction of the sample are
        given. To calculate cross sections, the relevant ``sample thickness'' is the areal
        density of nuclei $\rho_{\text{areal}}$, equivalent to
        the (volumetric) density times the length of the sample. For liquid
        samples H$_{2}^{\text{nat}}$O, D$_{2}^{\text{nat}}$O, and H$_{2}^{18}$O,
        the length and diameter listed are for the interior of the vessels
        used to hold the samples and the masses given are calculated based on 
        literature values for the density of each sample at 25 C.
        Our samples are generally much smaller than those used in previous
        measurements; for comparison, the Ni and Sn samples used in \cite{Abfalterer2001,
        Finlay1993} had areal densities of 1.515 and 0.5475
        $\frac{mol}{cm^{2}}$, respectively (12.7 and 6.5 times larger than our
        Ni and Sn samples). Columns 6 and 7 give the natural abundance of the
        isotope (NA) and the purity of our isotopic samples (SP).
    }
    \label{SampleCharacteristics}
    \begin{center}
        \begin{tabular}{ c c c c c c c }
            \hline
            Isotope & Len. [\milli\meter] & Diam. [\milli\meter]
            & Mass [\gram] & $\rho_{\text{areal}}$
            [$\frac{mol}{cm^{2}}$] & NA [\%] & SP 
            [\%]\\
            \hline

            $^{\text{nat}}$C & 13.66(2) & 8.260(5) & 1.2363
            & 0.1921(1) & - & -\\
            $^{\text{nat}}$C & 27.29(2) & 8.260(5) & 2.4680
            & 0.3835(2) & - & -\\
            \\
            H$_{2}$$^{\text{nat}}$O & 20.00(1) & 8.92(1) & 1.2461 & 0.1107(3) & - &
            - \\
            D$_{2}$$^{\text{nat}}$O & 20.00(1) & 8.92(1) & 1.3852 & 0.1107(3) &
            0.02 &
            99.9 \\
            H$_{2}$$^{18}$O & 20.00(1) & 8.92(1) & 1.3844 & 0.1107(3) & 0.20 &
            99.9\\
            \\
            $^{58}$Ni & 7.97(3)& 8.18(2) &
            3.6438 & 0.1197(3)& 68.1 & 99.6 \\
            $^{\text{nat}}$Ni & 8.00(3) & 8.20(2) &
            3.6898 & 0.1192(3)& - & -\\
            $^{64}$Ni & 7.96(2) & 8.20(4) &
            3.9942 & 0.1192(6) & 0.93 & 92.2\\
            \\
            $^{103}$Rh & 2.03(1) & 10.20(2) & 2.8359 & 0.02426(4) & 100 & 99.9\\
            \\
            $^{112}$Sn & 13.65(3) & 8.245(5) &
            4.9720 & 0.08332(5) & 0.97 & 99.9\\
            $^{\text{nat}}$Sn & 13.68(3) & 8.245(5) &
            5.3263 & 0.08414(5) & - & -\\
            $^{124}$Sn & 13.73(3) & 8.245(5) &
            5.5492 & 0.08399(5) & 5.79 & 99.9\\
            \\
            $^{\text{nat}}$Pb & 10.07(2) & 8.27(1) & 6.130 &
            0.05508(6) & - & -\\
            \hline
        \end{tabular}
    \end{center}
\end{table*}

%\begin{table*}[ht]
%    \caption{
%        Physical characteristics of all samples used for our \tot\
%        measurements. For isotopically-enriched samples, the natural abundance
%        of the enriched isotope and the isotopic fraction of the sample are
%        given. To calculate cross sections, the relevant ``sample thickness'' is the areal
%        density of nuclei $\rho_{\text{areal}}$, equivalent to
%        the (volumetric) density times the length of the sample. For liquid
%        samples H$_{2}^{\text{nat}}$O, D$_{2}^{\text{nat}}$O, and H$_{2}^{18}$O,
%        the length and diameter listed are for the interior of the vessels
%        used to hold the samples and the masses given are calculated based on 
%        literature values for the density of each sample at 25\textdegree{}C.
%        Our samples are generally much smaller than those used in previous
%        measurements; for comparison, the Ni and Sn samples used in \cite{Abfalterer2001,
%        Finlay1993} had areal densities of 1.515 and 0.5475
%        $\frac{mol}{cm^{2}}$, respectively (12.7 and 6.5 times larger than our
%    Ni and Sn samples).}
%    \label{SampleTable}
%    \begin{center}
%        \begin{tabular}{ c c c c c c c }
%            \hline
%            Isotope & Length & Diameter
%            & Mass & $\rho_{\text{areal}}$ & Nat. Abund. & Isotopic Purity\\
%                 & [mm] & [mm] & [g] & [$\frac{mol}{cm^{2}}$] & [\%] & [\%]\\
%            \hline
%
%            $^{\text{nat}}$C & 13.66(2) & 8.260(5) & 1.2363
%            & 0.1921(1) & - & -\\
%            $^{\text{nat}}$C & 27.29(2) & 8.260(5) & 2.4680
%            & 0.3835(2) & - & -\\
%
%            H$_{2}$$^{\text{nat}}$O & 20.00(1) & 8.92(1) & 1.2461 & 0.1107(3) & - &
%            - \\
%            D$_{2}$$^{\text{nat}}$O & 20.00(1) & 8.92(1) & 1.3852 & 0.1107(3) & - &
%            99.9 \\
%            H$_{2}$$^{18}$O & 20.00(1) & 8.92(1) & 1.3844 & 0.1107(3) & 0.20 &
%            99.0\\
%
%            $^{58}$Ni & 7.97(3)& 8.18(2) &
%            3.6438 & 0.1197(3)& 68.1 & 99.6 \\
%            $^{\text{nat}}$Ni & 8.00(3) & 8.20(2) &
%            3.6898 & 0.1192(3)& - & -\\
%            $^{64}$Ni & 7.96(2) & 8.20(4) &
%            3.9942 & 0.1192(6) & 0.93 & 92.2\\
%
%            $^{103}$Rh & 2.03(1) & 10.20(2) & 2.8359 & 0.02426(4) & 100 & 99.9\\
%
%            $^{112}$Sn & 13.65(3) & 8.245(5) &
%            4.9720 & 0.08332(5) & 0.97 & 99.9\\
%            $^{\text{nat}}$Sn & 13.68(3) & 8.245(5) &
%            5.3263 & 0.08414(5) & - & -\\
%            $^{124}$Sn & 13.73(3) & 8.245(5) &
%            5.5492 & 0.08399(5) & 5.79 & 99.9\\
%
%            $^{\text{nat}}$Pb & 10.07(2) & 8.27(1) & 6.130 &
%            0.05508(6) & - & -\\
%
%            \hline
%        \end{tabular}
%    \end{center}
%\end{table*}

The oxygen isotopes were prepared as water samples to increase the areal density
of atoms and for ease of handling. Each water sample was contained by a
cylindrical brass vessel with thin (0.002 inch) brass endcaps. Oxygen cross sections were calculated by
subtracting the well-known hydrogen cross section from the raw H$_{2}$O result
(we used H \tot\  data sets from Clement et al. \cite{Clement1972} and Abfalterer
et al. \cite{Abfalterer2001}, which together cover the range $0.5 \leq E_n \leq 500$ MeV
and are in excellent agreement where their energy ranges overlap). In light of
the additional uncertainty inherent to this kind of subtractive \tot\ determination,
a D$_{2}^{\text{nat}}$O sample from which the literature \tot\  for
D$_{2}$ could be subtracted was prepared as an additional cross-check. Due to
rhodium's poor machining properties, the \rhThree\ sample
was prepared by purchasing and stacking a series of thin discs rather than by
manufacturing a fused cylinder. These discs were corralled
by a cylindrical plastic case with open ends.

The sample configuration for each run varied, but generally all six positions on
the sample changer were used. For the solid targets, a typical configuration was
to place an empty styrofoam sample sleeve in the first sample-changer cradle as
the ``blank", the $^{\text{nat}}$C and $^{\text{nat}}$Pb samples in the second and third
cradles, and the samples of interest (e.g., $^{58}$Ni, $^{\text{nat}}$Ni, $^{64}$Ni) in
the fourth, fifth, and sixth cradles. For water samples, an empty brass vessel
was placed in the first cradle to serve as the blank.

\begin{figure}
    \includegraphics[width=0.3\textwidth]{figures/AllIsotopicSamples.jpg}
    \caption{(Color online) Section (a): the ${^{112,nat,124}}$Sn samples. Section (b): the 
        vessels used to hold water samples for the ${^{nat, 18}}$O \tot\  measurement. 
        Section (c): ${^{58,nat,64}}$Ni samples, shown end-on.}
    \label{SamplesImage}
\end{figure}

\section{Analysis}

The fundamental quantity of interest, \tot, is related to the flux
loss through a sample by:
\begin{equation}
I_{t} = I_{0}e^{-{\ell\rho\sigma_{tot}}}
\end{equation}
or, equivalently,
\begin{equation}
    \tot = -\frac{1}{\ell\rho}ln\left(\frac{I_{t}}{I_{0}}\right)
\end{equation}
where $I_{0}$ is the neutron flux entering the sample, $I_{t}$ is the neutron
flux transmitted through the sample without interaction, $\rho$ is the number
density of nuclei in the sample, and $\ell$ is the sample length. For thin
or low-density samples, flux attenuation through the sample will be small
(e.g., 13\% for our Ni samples at 100 MeV) and a large number
of counts will be required to determine the cross section to high
precision.

To identify valid neutron events and precisely determine the TOF start (micropulse start 
time) and TOF stop time (arrival at TOF detector) for each event, a series of corrections 
are required.  First, each event was assigned to the correct macropulse.
Time offsets accounting for cable and
electronics delay were applied, coarsely synchronizing all detectors with
facility-provided pulses indicating the arrival time of proton micropulses on the
tungsten target, so-called ``\tZero" pulses.
To improve the time resolution for each TOF
event, the digitized waveform for each event was passed 
through an offline software constant-fraction discriminator (CFD) algorithm
and a $\gamma$-ray-averaging
procedure (cf. \cite{Shane2010}) was used to improve the precision of the start 
time of each micropulse.  After these corrections, the final TOF resolution
(taken as the FWHM of the $\gamma$-ray peak in the TOF spectra) ranged from
0.60-0.90 ns over the series of \tot\ measurements.
This is comparable to the resolution from 
our digitizer-mediated \tot\ measurement on Ca isotopes in 2008 \cite{Shane2010}.
For context, for a 100-MeV neutron and a TOF detector distance of 27 meters, a TOF 
uncertainty of 0.80 ns translates to an energy resolution of $\approx$900 keV.
For neutrons below $\approx$20 MeV, the TOF time resolution worsens as the traversal time 
through the 1-inch thickness of the TOF detector becomes non-negligible.
However, because the TOF of these neutrons is already several hundred ns or
longer, the relative energy resolution ($\frac{\Delta E}{E}$) is
superior at low energies: for a 5 MeV neutron with a 0.82 ns detector-traversal time and
an inherent TOF resolution of 0.80 ns, $\Delta E$ is 13 keV. These energy uncertainties
have been propagated through subsequent analysis into our \tot\ results below.

Calculating the neutron energy requires knowledge of the flight path
distance to high precision. We determined this distance by calculating 
putative \tot\ data for $^{\text{nat}}$C from 3-15 MeV from our measurement and 
comparing the resonance peaks in this region with high-precision literature data
sets. The results of this analysis are shown in Fig. \ref{DistanceStudy};
the TOF distance was determined as 2709 $\pm$1 cm for the Ni and Rh
run configuration and 2554 $\pm$1 cm for the Sn and O run configuration.

\begin{figure}
    \includegraphics[width=0.5\textwidth]{figures/TimeCorrections.png}
    \caption{(Color online) The effects of timing corrections on the $\gamma$-ray
        peak of a typical run are shown. The uncorrected spectrum is shown in black,
        the spectrum after correction with our software CFD is shown in blue,
        and the spectrum after correction with both our software CFD and
        $\gamma$-averaging is 
        shown in pink. For this run, the final $\gamma$-ray peak 
        FWHM after both corrections is 0.866 ns, comparable to the precision we
        achieved in our Ca study \cite{Shane2010}, which also employed $\gamma$-
        averaging.
        }
    \label{TimingCorrectionStudy}
\end{figure}

Before cross sections could be tabulated, the per-event deadtime had to be
modeled and corrected for. Because events are not processed
instantaneously, there is a brief period
after each trigger during which the digitizer is busy processing that trigger.
Any newly-arriving events in this period will be ignored,
privileging events arriving earlier and thus distorting
TOF spectra and resulting cross sections. This busy period is referred to as the
``analytic" or ``per-event" deadtime and can be corrected for according to standard 
techniques
\cite{Moore1980}. An additional complication is the possibility of flux
variation between micropulses. If there is no variation, the fraction of time
that the digitizer is dead for a given time bin $i$ can be calculated with a
simple formula, per Moore's analysis of rate-dependence of counting losses
\cite{Moore1980}:
\begin{equation}
    F_{i} = \sum^{N-1}_{j=0} R_{(i-j)\text{ mod N}}\times P_{j}
    \label{DeadtimeEquation}
\end{equation}
where $N$ is the number of time bins in the micropulse, $R_{x}$ is the rate of
detected events per micropulse in bin $x$, and $P_{j}$ is the probability that the
digitizer is still busy from a trigger $j$ bins ago.
Moore also provides a more advanced formula to generate the appropriate
dead-time correction in cases where the variation in beam flux 
is significant. However, an examination of our flux-per-micropulse data showed
very little flux variance across macropulses, except during the first 10\%
of the micropulses within each macropulse. In our final analysis, we discarded 
the first 40 micropulses of each 
macropulse and used the simpler Eq. \ref{DeadtimeEquation} to calculate the dead
time fraction.

To model the experimentally-observed $P_{j}$, we
employed a logistic function and fitted it to the observed spectrum for time
differences between consecutive events (see Fig.
\ref{TimeDifferenceBetweenEvents}). For a given bin $i$, the fraction of time that the 
digitizer is dead, $F_{i}$, is in essence a discrete convolution of the
\textit{measured} TOF spectrum with $P_{j}$. Note that except for the first and
last micropulses in a macropulse, micropulses are consecutive and thus deadtime effects can
``wrap around" from the end of one micropulse to the next. For these wrap-around
contributions (that is, $j>i$), the (mod N) term ensures that the bin referred
to by $i-j$ is non-negative and has physical meaning as a time bin from the 
previous micropulse.

\begin{figure}
    \includegraphics[width=0.5\textwidth]{figures/TimeDifferenceBetweenEvents.png}
    \caption{(Color online) The time difference between adjacent TOF-detector
    events for a single run is plotted (black histogram). Below a certain
minimum time difference (the ``deadtime"), no events are recorded. A logistic
fit (red) models the detector's deadtime response and is used to generate a
deadtime correction. The underlying linearly-decreasing count rate (gray dashed
line) in incorporated into the logistic model. From the fit, a mean deadtime of
228.1 ns was extracted for the Sn and O run configurations (a similar
procedure was used to recover a deadtime of 159.7 ns for the Ni and Rh
run configurations).}
    \label{TimeDifferenceBetweenEvents}
\end{figure}

Because trigger processing is done in firmware onboard the digitizer,
the per-event deadtimes affecting our
measurement ranged from 150-230 ns. For a 230 ns deadtime, the probability-dead
$F_{i}$ of our time bins is given in Fig.
\ref{ExampleDeadtimeSpectrum}.
Once the fraction dead was identified for each time bin, the total number of
events detected in that bin, $N_{d}[i]$, was corrected to the \textit{true}
number of events in that bin $N_{t}[i]$ that would have been detected in the
absence of a
per-event deadtime:
\begin{equation}
    N_{t}[i] = -ln\left[1-\frac{\frac{N_{d}[i]}{M}}{(1-F_{i})}\right]\times M
\end{equation}
where M is the total number of micropulse periods. The difference between
uncorrected and analytic-deadtime-corrected TOF spectra is shown in Fig.
\ref{CorrectionEffectOnTOF}. At large TOFs (low energies) the correction is as low as a
few percent, but at small TOFs (high energies) when the digitizer is still dead
from the $\gamma$-ray flash and high-energy neutrons, the correction is significant
($\approx$20\% for our Ni/Rh runs, and $\approx$40\% for our Sn/O runs). These 
corrections are themselves far lower than the typical
analytic deadtime correction required with the deadtime mitigation scheme of
previous analog measurements \cite{Finlay1993,
Abfalterer2001}. %The effect of this correction on the final 
%cross section results is shown in Fig. \ref{DeadtimeEffectOnCS}.

In addition to analytic deadtime, there is an additional deadtime factor associated with 
digitizer readout to the data acquisition computer (DAQ). During data
collection, each pair of digitizer channels shares a common buffer for storing events.
After several seconds of acquisition, the digitizer tally of total events would
reach reaching a user-defined threshold to begin readout, at which time the
acquitision was paused and buffer contents passed out to the DAQ. However,
because each buffer was independently read out to the DAQ, it is possible that buffers
could be emptied and readied for new acquisition at slightly different times
(10-40 ms apart). Such run-time interactions between the firmware and USB
traffic of the DAQ were difficult to characterize, but we estimate that they might cause a 
systematic error of a few tenths of one
percent in the number of macropulses seen by differing channels, depending on the user-defined 
threshold and the buffer size, which could contribute to the discrepancy at the
highest energies ($>$100 MeV) between our results and past analog-enabled
measurements. 

During analysis, it was noted that occasionally (1 in 400 macropulses), one or two 
adjacent macropulses would have an abnormally small number of flux monitor events or 
TOF events. The frequency of these ``data dropouts" was similar to the rate of
switching between DPP and waveform modes and we suspect it is related to edge
case behavior right before or after a mode switch. To mitigate this issue,
any macropulse that had less than 50\% of the average event rate in either the
flux monitor or TOF detector channel was ignored during cross section calculation.

After applying these corrections, the veto and integrated charge gates are applied to 
all events and surviving events are populated into TOF spectra (see Fig.
\ref{ExampleTOFSpectrum}). Next, room background was subtracted (responsible for 0.1\% to 
1\% of event rate, depending on energy) and spectra were mapped to the energy domain.

\begin{figure}
    \includegraphics[width=0.5\textwidth]{figures/exampleTOFSpectrum.png}
    \caption{(Color online) TOF spectra after analytic deadtime correction and
        veto and integrated charge gating for the blank sample (in
        red) and for the $^{\text{nat}}$C sample (in blue), from the Ni/Rh experiment.
        The $\gamma$-ray peak is visible as a sharp spike at 90 ns, followed by
        the highest-energy neutrons at 130 ns. The small spikes spaced 60 ns
        apart (visible before 90 ns and after 1500
        ns) are identified as $\gamma$-ray peaks from a low-level, continuous ``drip" 
        of protons onto the tungsten target caused by mistuning of the proton 
        buncher; their effect on the calculated cross sections is negligible.
    }
    \label{ExampleTOFSpectrum}
\end{figure}

From these energy spectra, the raw cross sections were calculated, bin-wise, as follows:
$$
\tot = -\frac{1}{\ell\rho_{n}}
\ln \left(\frac{I_{0}}{I_{s}}\times\frac{M_{s}}{M_{0}}\right)
$$
where $\frac{I_{0}}{I_{s}}$ is the ratio of counts in the energy spectra between 
the blank and sample, $\frac{M_{s}}{M_{0}}$ is the ratio of counts in the
monitor detector between the sample and blank (for flux normalization), $\ell$ is the length 
of the sample, and $\rho_{n}$ is the number density of atoms in the sample.

A series of small corrections were applied to the raw cross sections to produce
the final results. First, because the blank sample contains air and not vacuum,
the cross section of air must be added to each other sample's cross section (scaled by  
the ratio of areal densities of air in the blank and the sample of interest).
For the sample with the smallest areal density (Rh), this correction was 2 mb.
The cross section for $^{64}$Ni was corrected for the isotopic enrichment of our
sample (92.2\%) using our measured $^{\text{nat}}$Ni cross section. All other isotopes were 
sufficiently pure such that the impurity correction was negligible.

To validate our analysis, we first benchmarked our \tot\ measurements of natural samples
($^{\text{nat}}$C, $^{\text{nat}}$Ni, $^{\text{nat}}$Sn, and
$^{\text{nat}}$Pb) against the high-precision data sets on natural samples from Finlay
\cite{Finlay1993} and Abfalterer \cite{Abfalterer2001}, shown in Fig.
\ref{LiteratureBenchmarking}. Our natural sample results
are in excellent agreement with 
these previous results from 3-100 MeV and show slight deviation above 100 MeV (a
relative difference of up to 5\% at 300 MeV), suggesting a small systematic
error at high energies in one or both approaches when the instantaneous neutron
flux is highest. As an additional diagnostic, we compared 
both our natural carbon targets against each other (see Fig.
\ref{CarbonBenchmarking}) and found agreement within 1\% throughout the measured
energy domain.

\begin{figure}
    \includegraphics[width=0.5\textwidth]{figures/literatureBenchmarking.png}
    \caption{(Color online) A comparison of literature data (taken with analog
    techniques) and our results (signals processed with a digitizer, or "DSP")
    for natural C, Ni, Sn, and Pb. In panel a), the absolute cross sections are shown from
    3-500 MeV; in panel b), the relative differences between the literature data and
    our data are shown in percent. From 3-100 MeV, our data are fully consistent with the
    literature datasets but above 100 MeV, a difference arises, peaking at
    $\approx$5\% at 300 MeV.
b}
    \label{LiteratureBenchmarking}
\end{figure}

\section{Experimental Results}

Our absolute \tot\ results for all isotopic targets are shown in Figs.
\ref{TwoPanelO}, \ref{TwoPanelNi}, \ref{TwoPanelRh}, and \ref{TwoPanelSn}.
All previous
isotopic \tot\ measurements
(where they exist) are shown alongside our results for comparison.
To compare the results, relative differences between our data and literature
data are also plotted. The literature
data sets shown have been modified to have the same bins as our data via a simple
linear interpolation of the original data's bins. This rebinning
washes out the fine structure of the cross section data where the density of states
for a given sample is low and individual resonances are visible
(e.g., $^{\text{nat}}$C below 10 MeV).

Relative differences of our measured \tot\ between $^{16,18}$O, $^{58,64}$Ni, and
$^{112,124}$Sn are shown in Figs. \ref{IsotopicDifferenceO},
\ref{IsotopicDifferenceNi}, and 
\ref{IsotopicDifferenceSn}, respectively. 
\begin{figure}[tb]
    \centering
    \includegraphics[width=0.5\textwidth]{figures/TwoPanelO.png}
    \caption[Neutron \tot\ for \oSixEight: our results and literature data]
    {Neutron \tot\ for \oSixEight: our results and literature data.
        (a) shows our digitizer-measured isotopic results (in shades of red) and
        corresponding analog-measured literature data \cite{Finlay1993, Perey1972, Vaughn1965,
        Salisbury1965} (in shades of blue). The data for \oEight\ have been
        shifted up by 1 barn for visibility.
        (b) shows the relative difference between our data
        and literature data that are shown in panel (a).
    }
    \label{TwoPanelO}
\end{figure}

\begin{figure}[tb]
    \centering
    \includegraphics[width=0.5\textwidth]{figures/TwoPanelNi.png}
    \caption[Neutron \tot\ for \niEightFour: our results and literature data]
    {
        Neutron \tot\ for \niEightFour: our results and literature data.
        (a) shows our digitizer-measured isotopic results (in shades of red) and
        corresponding analog-measured literature data \cite{Perey1993,
        Dukarevich1967} (in shades of blue). (b) shows the relative difference
        between our data and literature data that are shown in panel a).
    }
    \label{TwoPanelNi}
\end{figure}
\begin{figure}[tb]
    \centering
    \includegraphics[width=0.5\textwidth]{figures/TwoPanelRh.png}
    \caption[Neutron \tot\ for \rhThree: our results and literature data]
    {
        Neutron \tot\ for \rhThree: our results and literature data.
        (a) shows our digitizer-measured isotopic results (in red) and
        corresponding analog-measured literature data \cite{Poenitz1983} (in blue).
        (b) shows the relative difference
        between our data and literature data that are shown in panel (a).
    }
    \label{TwoPanelRh}
\end{figure}
\begin{figure}[tb]
    \centering
    \includegraphics[width=0.5\textwidth]{figures/TwoPanelSn.png}
    \caption[Neutron \tot\ for \snTwelveFour: our results and literature data]
    {
        Neutron \tot\ for \snTwelveFour: our results and literature data.
        (a) shows our digitizer-measured isotopic results (in red) and
        corresponding analog-measured literature data \cite{Harper1982, Timokhov1989, 
        Rapaport1980, Dukarevich1967} (in shades of blue).
        (b) shows the relative difference between our data
        and literature data that are shown in panel (a). The data sets are in
        excellent agreement where literature data exist, up to 20 \mega\electronvolt.
    }
    \label{TwoPanelSn}
\end{figure}

\begin{figure}[tb]
    \centering
    \includegraphics[width=0.5\textwidth]{figures/relativeDiff_O18O16.png}
    \caption[\oSixEight\ neutron \tot\ relative difference]
    {\oSixEight\ neutron \tot\ relative difference, from our newly-measured
        data sets (in red). Assuming a simple A$^{\frac{1}{3}}$ size scaling for the
        nuclear radius of \oEight\ from \oSix, the strongly-absorbing-sphere model 
        Eq. \ref{SASAbsolute} prediction for the \oSixEight\ neutron \tot\ relative
        difference is shown (black dashed line). From 10-100 \mega\electronvolt, the SAS
        model is surprisingly successful at reproducing the relative difference,
        though it fails above 100 \mega\electronvolt\ as the \oEight\ neutron \tot\ drops below
    that of \oSix.}
    \label{IsotopicDifferenceO}
\end{figure}

\begin{figure}[tb]
    \centering
    \includegraphics[width=0.5\textwidth]{figures/relativeDiff_Ni64Ni58.png}
    \caption[\niEightFour\ neutron \tot\ relative difference]
    {
        \niEightFour\ neutron \tot\ relative difference, from our newly-measured
        data sets (in red). Total error (including statistical and systematic)
        is indicated by the red shaded region. Systematic error only (due to
        uncertainty in the sample areal density) is shown by the blue shaded region.   
        Assuming a simple A$^{\frac{1}{3}}$ size scaling for the
        nuclear radius of \niFour\ from \niEight, the simple
        strongly-absorbing-sphere model Eq. \ref{SASAbsolute} prediction
        for the \niEightFour\ neutron \tot\ relative
        difference is shown (black dashed line).
    }
    \label{IsotopicDifferenceNi}
\end{figure}

\begin{figure}[tb]
    \centering
    \includegraphics[width=0.5\textwidth]{figures/relativeDiff_Sn124Sn112.png}
    \caption[\snTwelveFour\ neutron \tot\ relative difference]
    {
        \snTwelveFour\ neutron \tot\ relative difference, from our newly-measured
        data sets (in red). Total error (including statistical and systematic)
        is indicated by the red shaded region. Systematic error only (due to
        uncertainty in the sample areal density) is shown by the blue shaded region.   
        The predictions of the simple strongly-absorbing-sphere model of
        Eq. \ref{SASAbsolute} for the \snTwelveFour\ neutron \tot\ relative
        difference are shown for two possible nuclear radius size scalings: $r \alpha A^{\frac{1}{3}}$
        (black dashed line) and $r \alpha A^{\frac{1}{6}}$ (gray dashed line).
    }
    \label{IsotopicDifferenceSn}
\end{figure}

\section{DOM Analysis}

The Dispersive Optical Model (DOM) is a phenomenological Green's-function
framework enabling a simultaneous and self-consistent analysis of nuclear
structure and reaction
data. An essential feature of the DOM is the enforcement of a dispersion
relation between the complex components of the self-energy across the entire
energy spectrum, allowing data from below the Fermi energy (charge densities,
bound levels) to constrain the potential above, and data from above the Fermi
energy (elastic, reaction, and total cross sections) to constrain the potential
below. Details on the DOM formalism and results of many related DOM analyses are available in
\cite{Mahaux1991, Dickhoff2018, PruittPhDThesis, AtkinsonPhDThesis,
MahzoonPhDThesis}. In this section we present pertinent findings from fits on \oSixEight,
\niEightFour, and \snTwelveFour. The analyses of \oSixEight\ are presented in
greater detail to illustrate our approach, which uses an updated version
of the DOM code that has been generalized for use with any doubly-even nucleus.
This version is the first to extend beyond doubly-closed-shell nuclei and
accommodates the partial filling
of open shells, as for neutrons in \oEight, with a simple pairing parameter
$\Delta$ (details are provided in \cite{PruittPhDThesis}).
The full DOM potential parametrization and the optimized parameter values
recovered from the fitting are provided in appendix A.

\subsection{Experimental data used to constrain $^{16}$O potential}

Nucleon elastic scattering data (including differential cross sections and analyzing
powers) from 10-200 MeV and nucleon reaction cross sections from 20-65 MeV
were retrieved from the EXFOR database \cite{EXFORDatabase}.
For protons, twenty-eight elastic differential cross
sections data sets, twenty analyzing power data sets,
and three reaction cross section data sets were
incorporated in this energy range. Due to the lack of experimental proton
reaction cross sections between 65-200 MeV, we generated proton reaction cross
section data points based on systematic trends in experimental data 
as identified in the comprehensive review of Carlson \cite{Carlson1996}.
These pseudo-data points were included in the fit. For
neutrons, ten neutron elastic differential cross section
data sets, ranging from a neutron energy of 10 MeV to 95 MeV, and a single
neutron reaction cross section data point, at 14 MeV, were included. 
Our newly-measured \tot\ results for $^{16}$O were included as well. In all,
over sixty experimental nucleon scattering data sets were used to constrain the
\oSix\ parameters.

In addition to nucleon scattering data, several sectors of bound-state data were
included in the fit. Neutron (proton) 0p$_{\frac{1}{2}}$ and 0d$_{\frac{5}{2}}$ energies were
assigned according to the mass differences between the $^{15,16}$O and
$^{16,17}$O isotopes ($^{15}$N, $^{16}$O; $^{16}$O, $^{17}$F isotopes) \cite{AME2016}.
The charge density profile and RMS charge radius used were extracted/compiled by
De Vries et al from elastic electron scattering data
\cite{DeVries1987}. To further constrain the shape of the imaginary
potential below the Fermi energy, the locations and widths of the spectral peaks
of the proton bound states (0s$_\frac{1}{2}$, 0p$_\frac{3}{2}$, and
0p$_\frac{1}{2}$) were fitted to the separation energies and hole widths determined
from quasi-free (e,e'p) and (p,2p) scattering \cite{Mougey1980, Jacob1966,
Jacob1973}. Lastly, the experimentally-known total binding energy was included. 

\subsection{Experimental data used to constrain $^{18}$O potential}

As with $^{16}$O, a large corpus of proton elastic scattering data was
available in the EXFOR database. Twenty-eight proton elastic differential cross
sections were include, with an energy range from 8-200 MeV in the lab frame.
No proton reaction cross sections were available. On the neutron side, three
elastic differential cross section data sets were used, at 8.5, 14, and 24 MeV.
Only one data point for the neutron reaction cross section was available (14.1
MeV) and was included. Our \tot\ results for $^{18}$O were the
sole neutron total cross section data used in the fit. The energies of the
neutron (proton) 0p$_{\frac{1}{2}}$ and 0d$_{\frac{5}{2}}$ orbitals were
assigned according to the mass differences between the $^{15,16}$O and
$^{16,17}$O isotopes ($^{15}$N, $^{16}$O; $^{16}$O, $^{17}$F isotopes) \cite{AME2016}.
The charge density profile and RMS charge radius were assigned according to
those extracted by DeVries et al from elastic electron scattering data
\cite{DeVries1987}. Lastly, to further constrain the strength of the imaginary
component of the potential below the Fermi energy, the spectral peak location and
width of each of the bound states (0s$_\frac{1}{2}$, 0p$_\frac{3}{2}$, and
0p$_\frac{1}{2}$) were fitted to the separation energy and hole widths determined
from (e,e'p) and (p,2p) quasi-free scattering \cite{Mougey1980, Jacob1966,
Jacob1973}.

\subsection{Fit results on $^{16}$O}
\subsection{Fit results on $^{18}$O}

\subsection{Discussion of fit results}

In a previous DOM analysis \cite{}, the neutron skin thickness of $^{48}$Ca was
found to be highly sensitive to the $^{40,48}$Ca \tot\ data used to constrain
the potential. 

\section{Conclusion}
Isotopically-resolved neutron total cross sections are a valuable probe of the
isovector component of optical potentials. By adopting a digitizer-driven
approach, we have measured \tot\ on the important closed-shell nuclides
$^{16,18}$O, $^{58,64}$Ni, and $^{112,124}$Sn across more than two orders of
magnitude in energy (3-450 MeV). Except at the highest energies, our results
on natural targets in are good agreement with previous analog measurements
that required 10-20 times more target material. 

\section{Acknowledgements}

\bibliography{references}
\begin{thebibliography}{32} \expandafter\ifx\csname
        natexlab\endcsname\relax\def\natexlab#1{#1}\fi \expandafter\ifx\csname
        bibnamefont\endcsname\relax \def\bibnamefont#1{#1}\fi
        \expandafter\ifx\csname bibfnamefont\endcsname\relax
        \def\bibfnamefont#1{#1}\fi \expandafter\ifx\csname
        citenamefont\endcsname\relax \def\citenamefont#1{#1}\fi
        \expandafter\ifx\csname url\endcsname\relax \def\url#1{\texttt{#1}}\fi
        \expandafter\ifx\csname urlprefix\endcsname\relax\def\urlprefix{URL
        }\fi \providecommand{\bibinfo}[2]{#2}
        \providecommand{\eprint}[2][]{\url{#2}}

\end{thebibliography}

\end{document}
