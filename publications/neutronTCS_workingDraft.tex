\documentclass[twocolumn,secnumarabic,amssymb, nobibnotes, aps, prl,
superscriptaddress, nobalancelastpage]{revtex4}
\newcommand{\revtex}{REV\TeX\ }
\newcommand{\classoption}[1]{\texttt{#1}}
\newcommand{\macro}[1]{\texttt{\textbackslash#1}}
\newcommand{\m}[1]{\macro{#1}} \newcommand{\env}[1]{\texttt{#1}}
\setlength{\textheight}{9.5in} \usepackage{graphicx}
\setlength{\belowcaptionskip}{6pt}
\usepackage{amsmath} \usepackage{braket} \usepackage{epsfig}
\usepackage{upgreek}

\newcommand{\tot}{\ensuremath{\sigma_{tot}}}
\newcommand{\tots}{\ensuremath{\sigma_{tot}}\,\,}
\newcommand{\totE}{\ensuremath{\sigma_{tot}}(E)}
\newcommand{\totEs}{\ensuremath{\sigma_{tot}}(E)\,\,}
\newcommand{\totRDs}{\ensuremath{\sigma_{A,A'}}(E)\,}

\begin{document}

\begin{abstract}
    The neutron total cross sections \tot of $^{16,18}$O,
    $^{58,64}$Ni, $^{103}$Rh, and $^{112,124}$Sn have been measured at the Los Alamos
    Neutron Science Center (LANSCE) at intermediate energies (3 $\leq E_{n}
    \leq$ 500 MeV) by
    leveraging waveform digitizer technology. The results are in good agreement
    with previous measurements that used analog techniques on natural targets,
    excepting small deviations at high energies, and continue the campaign of
    \tots measurements we initiated with the case study of $^{40,48}$Ca in 2009.
    The \tots relative differences between isotopes (e.g.,
    $\frac{\sigma_{Ni^{64}}-\sigma_{Ni^{58}}} {\sigma_{Ni^{64}}+
    \sigma_{Ni^{58}}}$)
    depart from the isoscalar picture and reveal additional information about
    the isovector components needed for an accurate optical model (OM)
    description away from stability. Digitizer-enabled \tot-measurement
    techniques are discussed and a preliminary Dispersive Optical Model (DOM)
    analysis using these data is presented.
\end{abstract}

\title{Isotopically-Resolved Neutron Total Cross Sections At
Intermediate Energies}

\author{C.~D.~Pruitt}  \email[Corresponding author:]{cdpruitt@wustl.edu}
\author{R.~J.~Charity}
\author{D. E.~M.~Hoff}  
\author{L.~G.~Sobotka}
\author{K.~W.~Brown} \altaffiliation{Present Address: \textit{National
        Superconducting Cyclotron Laboratory, Departments of Physics and
Astronomy, Michigan State University, East Lansing, MI 48824, USA}}
\author{J.~M.~Elson}
\affiliation{Department of Chemistry, Washington University, St. Louis, MO 63130}

\author{H. Y. Lee}
\author{M. Devlin}
\author{N. Fotiadis}
\author{S. Mosby}
\affiliation{Los Alamos National Lab, Los Alamos, NM 87545, USA}
\maketitle

Neutron scattering is a direct, Coulomb-insensitive tool for probing the nuclear
environment. The simplest measurement of neutron interaction with a nucleus,
the neutron total cross section \totE, provides fundamental information about
nuclear size and the ratio of elastic-to-inelastic components of nucleon 
scattering. Additionally, \tots data are sensitive to a variey of nuclear
properties of great interest including the neutron skin of neutron-rich nuclei
\cite{Mahzoon2017} and thus the density dependence of the symmetry energy $L$,
essential for an accurate neutron star equation-of-state (EOS)
\cite{Fattoyev2012, Vinas2014, Brown2000}.

The earliest model for neutron scattering (a ``strongly-absorbing sphere"
picture) describes the nucleus as a constant-density sphere that interacts
strongly with incident neutrons approaching within a nuclear radius
\cite{Feshbach1949}. In this picture devoid of nuclear structure, \totEs depends
only on size scaling of the interacting bodies:
\begin{equation} \label{eq1}
    \sigma_{tot}(E) = 2\pi(R + \lambdabar)^{2}
\end{equation}
where $R=r_{0}A^{\frac{1}{3}}$ and $\lambdabar$ is the reduced wavelength
of the incident neutron in the center of mass \cite{Fernbach1949, Satchler1980}. 
To compare \totEs for two different targets of masses A and A', the relative
difference \totRDs is useful:
\begin{equation} \label{eq2}
    \begin{split}
        \totRDs & \equiv
    \frac{\sigma_{A}-\sigma_{A'}}{\sigma_{A}+\sigma_{A'}} \\
    & =
        \frac{r_{0}(A^{\frac{2}{3}}-A'^{\frac{2}{3}}) +
        2\lambdabar(A^{\frac{1}{3}}-A'^{\frac{1}{3}})}
        {r_{0}(A^{\frac{2}{3}}+A'^{\frac{2}{3}}) +
        2\lambdabar(A^{\frac{1}{3}}+A'^{\frac{1}{3}})}
    \end{split}
\end{equation}
While on \textit{average}, experimental data comport with this naive
model, the key feature of \totEs data is the obvious oscillatory
behavior centered about the average (visible in Fig.
\ref{SASphereVsExperiment}).

\begin{figure*}
    \includegraphics[scale=0.3]{../analysis/plots/theory/SASphereVsExperiment.png}
    \caption{(Color online) Experimental \totEs data are shown from 2-500
        MeV for nuclides from A=12 to A=208
        \cite{Finlay1993, Schwartz1974, Poenitz1983, Abfalterer2000, Abfalterer2001}.
        Predictions for \totEs given by the "strongly-absorbing sphere" (SAS) model (equation 
        \ref{eq1}) are shown as thin dashed lines for each nucleus.
        Regular oscillations about the SAS model are clearly visible
        as is the trend for the oscillation
        maxima/minima to shift to \textit{higher} energies as A is increased. At low energies 
        resonance structures are visible especially for light nuclides where the
        density of states is smallest. Note that at higher energies, the experimental
        cross sections drop below those predicted by the SAS model, illustrating
    the failure of the SAS model to describe [surface physics accounted for in
optical models?].}
    \label{SASphereVsExperiment}
\end{figure*}

%\begin{equation} \label{eq2}
%    \sigma_{tot}(E,A) \propto \frac{A^{\frac{2}{3}}}{E}
%    \times sin\left(\frac{A^{\frac{1}{3}}}{E^{\frac{1}{2}}}\right)
%\end{equation}
%that also appears in the relative difference:
%\begin{equation} \label{eq2}
%    \frac{\sigma_{A}-\sigma_{A'}}{\sigma_{A}+\sigma_{A'}} =
%    \frac
%    {A^{\frac{2}{3}}sin(\frac{A^{\frac{1}{3}}}{E^{\frac{1}{2}}}) -
%    A'^{\frac{2}{3}}sin(\frac{A'^{\frac{1}{3}}}{E^{\frac{1}{2}}})}
%    {A^{\frac{2}{3}}sin(\frac{A^{\frac{1}{3}}}{E^{\frac{1}{2}}}) +
%    A'^{\frac{2}{3}}sin(\frac{A'^{\frac{1}{3}}}{E^{\frac{1}{2}}})}
%\end{equation}

These oscillations can be explained as the result of a phase shift between 
neutron waves passing around the nucleus (unshifted) and waves passing
through the the nucleus, where they experience refraction
(illustrated in figure \ref{RamsauerPhaseShiftFigure}). This explanation was termed the ``nuclear 
Ramsauer effect" by Peterson \cite{Peterson1962}, based on the analagous effect seen in 
electron scattering on noble gases.
In this picture, the intractable many-body
problem of the target nucleus are replaced by a smooth potential (hence the
``optical model" of the nucleus, \cite{Feshbach1958}) from which the refractive
index is easily calculated. As in the optical case, a complex index of
refraction can be used to neatly incorporate both elastic scattering (the real
potential) and inelastic scattering (the imaginary potential).
%\cite{McVoy1967}. 

\begin{figure}
    \hfill\begin{minipage}{0.5\textwidth}\centering
        \includegraphics[scale=0.2]{figures/phaseShiftStillsFigure.png}
        \caption{(Color online) Frames taken from an simulation illustrating the nuclear 
            Ramsauer effect. A neutron wave train (series of
            blue lines) impinges from the left on a real Woods-Saxon
            potential centered at the origin (diffuse red circle). The potential
            refracts the neutron wave,
            retarding the phase of the wavefront as it passes through the
            potential. After escaping the potential, a phase difference $\Delta$ between
            the wave component passing \textit{around} and \textit{through the center}
            of the potential persists, resulting in scattering.
            For the leading wavefront in the wave train, $\Delta$ is indicated in
            the top right-hand corner of each panel. A differential version of equation 
            \ref{phaseShift} is used to
            calculate the phase shift for each step. In this figure, the neutron
            energy $E_{n}$ = 14 MeV and nuclear mass $A$ = 25. For the Woods-Saxon potential,
            we used a potential depth $U$ = 42.8 MeV (following Angeli's analysis
            of \tots data at 14 MeV \cite{Angeli1970}), with nuclear radius $R = 
            r_{0}A^{\frac{1}{3}}$, $r_{0}$ = 1.4 fm, and a diffuseness parameter
        $a$ of 0.5.}
        \label{RamsauerPhaseShiftFigure}
    \end{minipage}
\end{figure}


Following Angeli \cite{Angeli1970}, these considerations can be incorporated by
imbuing the strongly-absorbing sphere relations (equation \ref{eq1}) with an additional sinusoidal term:
\begin{equation}
    \totE = 2\pi (R+\lambdabar)[1 - \rho \cos(\delta)]
\end{equation}
where $\rho = e^{-\operatorname{Im}(\Delta)}$, and $\delta =
\operatorname{Re}(\Delta)$, with $\Delta$ the phase difference between the wave traveling
around and traveling through the nucleus. Thus, the amplitude of the oscillations provides the 
inelastic phase shift and the period of oscillation provides the elastic phase shift.
As can be seen from equation 
\ref{phaseShift}, the large magnitude of the oscillations means that inelastic
scattering accounts for only a small fraction of the total cross section, in turn implying a 
much larger mean free path for neutrons through the nucleus 
than might otherwise be expected (i.e., in the absence of Pauli blocking)
\cite{Mohr1955}.

If we approximate the nucleus with a
spherical potential of radius $R$ and depth $U$, the total phase shift $\delta$ is:
\begin{equation} \label{phaseShift}
    \delta =
    \frac{\overline{C}\left(\left[{\frac{E+U}{E}}\right]^{\frac{1}{2}}-1\right)}{\lambdabar}
\end{equation}
where $\overline{C} = \frac{4}{3}R$ is the average chord length through the
sphere \cite{Angeli1970}. Plotting $\delta$ as a function of E and A also reveals an important
relation: to maintain constant phase as A is increased, E must also increase 
\cite{Satchler1980, Peterson1962}:
\begin{equation}
    \delta \propto A^{\frac{1}{3}}\times\left(\sqrt{E+U}-\sqrt{E}\right)
\end{equation}
This is contrary to a typical resonance condition where an integer number of wavelengths
are fit inside a potential, where to maintain constant phase as A is increased,
E must be decreased. Thus these \tots oscillations have been referred to as
``anti-resonances" or ``echoes" \cite{Satchler1980, McVoy1967}.

By including additional surface, spin-orbit, and other terms, optical models (OMs) have been 
used to successfully reproduce the general features of all manner of single-nucleon scattering 
data across the chart of nuclides up to several hundred MeV [insert citation].
However, despite the excellent agreement with experiment, optical models
involve the interaction of many partial waves with many sometimes-opaque terms
in the potential, complicating intuitive understanding of the underlying
physics at work.

With these considerations in mind, our goal is twofold: to demonstrate the
value of isotopically-resolved \totEs data for identifying the dependence of optical 
potential terms on nuclear asymmetry, and, using a Dispersive Optical Model
(DOM) analysis of these \totEs data, to extract important quantities (neutron skin
thicknesses, spectroscopic factors) highly sensitive to asymmetry.
As one moves further from N=Z, these isovector components of the optical potential
are increasingly important but also increasingly
difficult to constrain as they requires both proton and neutron scattering data. Experimental 
challenges inherent to neutron scattering have meant that these data are often 
limited or missing entirely despite their usefulness, motivating our \totEs
measurement program.

%from  Anderson and Grimes were the first to define an isospin-isospin cross section
%$\sigma_{is}$:
%\begin{equation}
%    \sigma = \frac{\sigma_{tot}^{>} - \sigma_{tot}^{<}}{2}
%\end{equation}
%where > and < refer to the isospin parallel ($T_0 + \frac{1}{2}$) and
%anti-parallel ($T_0 - \frac{1}{2}$) projections, with the explicit intention of
%examining the differences in \totEs between isotopes to learn more about the
%isovector components of the optical potentials \cite{Anderson1990}.

\section{Experimental Considerations}
By scattering secondary radioactive beams off of hydrogen targets in inverse
kinematics, proton-scattering experiments are possible even on highly unstable
nuclides. In contrast, because neutrons themselves must be generated as a
secondary radioactive beam, neutron-scattering experiments are restricted to
normal kinematics and \tots measurements are possible only for relatively stable
nuclides that can be formed into a target. At present, \tots measurements above
the resonance region on nuclides short half-lives (shorter than the timescale of
days) are technically infeasible for this reason, though a handful have been carried out on
samples with half-lives in the tens to thousands of years \cite{Poenitz1983,
Phillips1980, Foster1971}.

Traditionally, \tots measurements have relied on analog techniques for recording
events, techniques that suffer from a large per-event or ``analytic" deadtime of
up to several $\upmu$s. Thus for a typical intermediate-energy \totEs measurement
with dozens or hundreds of energy bins, achieving statistical uncertainty at the
level of 1\% requires a thick sample to attenuate a sizable fraction of the
incident neutron flux. For cross sections in the 1-10 barn range, this means
sample masses of tens of grams \cite{Finlay1993, Abfalterer2001}.
Producing an isotopically-enriched sample of this size is often
prohibitively expensive; indeed, a literature search for isotopically-resolved
\tots measurements reveals a paucity of data from 1-300 MeV, even for
closed-shell isotopes of special importance like $^{3,4}$He, $^{64}$Ni, and
$^{204}$Pb (see Table \ref{IsotopicCrossSectionTable}).

\begin{table}[ht]
    \caption{A selection of results from a literature study of isotopically-
    resolved \totEs data using the EXFOR database \cite{EXFORDatabase}. For the
    heaviest and lightest stable nuclides in each closed shell in Z, all
    datasets falling at least partially within 1-500 MeV are shown. For elements
    whose natural abundance is $>$90\% of a single isotope (e.g.,
    96.9\% of $^{\text{nat}}$Ca is $^{40}$Ca), \totEs data on the natural
    target was includedas as ``isotopic".} \label{tab:title}
    \label{IsotopicCrossSectionTable}
    \begin{center}
        \begin{tabular}{ c c c c }
            \hline
            Isotope & Natural Abundance & Energy Range (MeV) & Reference\\
            \hline

            $^{3}$He & $2\times 10^{-4}\%$ & $1.5 - 40$ & \cite{Haesner1983}\\
            $^{4}$He & $>99.9\%$ & $0.7-30$ & \cite{Goulding1973}\\
            & & $2-40$ & \cite{Haesner1983}\\
            & & $77-151$ & \cite{Measday1966}\\

            $^{16}$O & $99.8\%$ & $0.2-49$ & \cite{Perey1972}\\
            & & $5-600$ & \cite{Finlay1993}\\

            $^{18}$O & $0.20\%$ & $0.1-2.5$ & \cite{Vaughn1965}\\
            & & $2.5-19$ & \cite{Salisbury1965}\\

            $^{40}$Ca & $96.9\%$ & $<0.1-6.4$ & \cite{Johnson1973}\\
            & & $5.3-560$ & \cite{Abfalterer2001}\\

            $^{48}$Ca & $0.187\%$ & $0.6-5.2$ & \cite{Harvey1985}\\
            & & $12-276$ & \cite{Shane2010}\\

            $^{58}$Ni & $68.1\%$ & $<0.1-68$ & \cite{Perey1993}\\

            $^{64}$Ni & $0.926\%$ & $14.1$ & \cite{Dukarevich1967}\\

            $^{112}$Sn & $0.97\%$ & $<0.1-1.4$ & \cite{Timokhov1989}\\
            & & $14.1$ & \cite{Dukarevich1967}\\

            $^{124}$Sn & $5.79\%$ & $0.3-5.0$ & \cite{Harper1982}\\
            & & $5.1-26$ & \cite{Rapaport1980}\\

            $^{204}$Pb & $1.4\%$ & $<0.1-27$ & \cite{Carlton2003}\\

            $^{208}$Pb & $52.4\%$ & $<0.1 - 695$ & \cite{Harvey1999}\\
            & & $5-600$ & \cite{Finlay1993}\\

            \hline
        \end{tabular}
    \end{center}
\end{table}

Recent developments in waveform digitizer technology, however, have made it
possible to reduce the per-event deadtime by an order of magnitude or more,
enabling a corresponding reduction in the necessary sample size. Thus in 2008 we
embarked on a campaign of \tots measurements on isotopically-enriched samples,
starting with $^{40,48}$Ca from $15 \leq E_{n} \leq 300$ MeV \cite{Shane2010}.
The data from that measurement have been incorporated into a comprehensive
Dispersive Optical Model (DOM) analysis \cite{Mueller2011, Mahzoon2014,
MahzoonPhDThesis} yielding proton and neutron spectroscopic factors, charge
radii, and neutron skins \cite{Mahzoon2017} for these nuclei. To continue our
program, we now preesnt \tots results for the important closed-shell nuclides
$^{16,18}$O, $^{58,64}$Ni, and $^{112,124}$Sn and the naturally-monoisotopic
$^{103}$Rh and describe further developments in the digitizer-enabled
approach.

\section{Experimental Details}
All neutron total cross section measurements were carried out at the 15R
beamline at the Weapons Neutron Research (WNR) facility of the Los Alamos
Neutron Science Center (LANSCE) during two run cycles (November 2016 and
September 2017). Our experiment was modeled on previous
\tots measurements at WNR \cite{Finlay1993,Abfalterer2001,Shane2010}. At WNR,
broad-spectrum neutrons up
to 800 MeV are generated by impinging proton pulses onto a water-cooled, 7.5
cm-long tungsten target (see Fig. \ref{ExperimentalApparatus}). Before the beam
enters the experimental area, a
permanent magnet deflects all charged particles generated by the proton pulses, 
allowing only neutrons and gamma rays to reach the flight path. At the
entrance to the flight path, the beam was collimated to 0.200 inches using steel
donuts with a total thickness of 24 inches and hardened using a plug of Hevimet (90\% W, 6\% 
Ni, 4\% Cu by weight) inserted at the upstream entrance of the
collimation stack. After collimation, the beam passed successively through a flux 
monitor, the sample of interest held in a sample changer, a veto detector, and finally the 
time-of-flight (TOF) detector approximately 25 meters from the neutron source.
All detectors consisted of BC-400 fast scintillating plastic mated with 
photomultiplier tubes (PMTs) and encased in a plastic or
aluminium structural housing. The flux monitor and veto detector each had
plastic thicknesses of $\frac{1}{4}$ inch and the TOF detector had a plastic
thickness of 1 inch. Signals from all detectors and
the target changer were relayed to a 500-MHz CAEN DT-5730 waveform digitizer
running custom software. To improve time resolution, the TOF detector used two
PMTs (one left, one right) mated to the plastic scintillator and the PMTs' signals were 
summed before digitization.

\begin{figure}
    \includegraphics[scale=0.6]{figures/ExperimentalSetup.png}
    \caption{(Color online) Experimental configuration at WNR facility. After a
        permanent magnet sweeps charged particles from the beam, neutrons and
        $\gamma$ rays are collimated to a 0.200 inch beam en route to the
        detectors used in the experiment. Samples are cycled into and out of beam
        using a linear actuator with a period of 150 seconds. Times-of-flight (TOFs) are
    determined by the TOF detector and used to calculate neutron energy.}
    \label{ExperimentalApparatus}
\end{figure}

\begin{figure*}
    \includegraphics[scale=0.4]{figures/beamStructure.png}
    \caption{(Color online) Neutron beam structure at WNR facility.
        ``Macropulses" of protons (row d) are delivered to
        WNR's tungsten Target 4, where they generate neutrons by spallation.
        Each macropulse consists of
        $\approx$ 350 proton ``micropulses" (row c). Neutrons
        from each micropulse (row b) disperse in
        time as they travel along the flight path so that $\gamma$ rays and high-energy 
    neutrons catch up to low-energy ones from the previous pulse (row a).}
    \label{BeamStructure}
\end{figure*}

The particular neutron beam structure at WNR dictates the energy range
achievable for \totEs measurements (see Fig. \ref{BeamStructure}).
Proton pulse trains, called ``macropulses", are delivered to the tungsten target at 120 Hz.
Each macropulse consists of ~350 individual proton pulses, called ``micropulses", spaced 1.8 
$\upmu$s apart. Each micropulse consists of a single proton packet $<$1 ns wide when it 
arrives at the tungsten target that generates gamma rays and neutrons within a tight
temporal-spatial range. As neutrons from this micropulse travel along the beam path, 
high energy neutrons separate in time from lower-energy neutrons so that neutron
energy can be determined by standard TOF techniques (see \cite{Moore1980} for details).
Because the $\gamma$-rays and high-energy neutrons from later micropulses can
overtake slower neutrons from an earlier micropulse, the distance of the TOF
detector from the neutron source determines both the minimum neutron energy that can be 
unambiguously resolved and the maximum instantaneous neutron flux, critical to correcting
for per-event deadtime.

A programmable sample changer with six positions
was used to cycle each sample into the beam at a regular interval of 150 seconds 
per sample. Once per macropulse, an analog signal from the sample changer was recorded to 
indicate its current position. Variations in beam flux 
between macropulses were monitored by the flux monitor detector. To account for
charged-particle production in the samples and in air along the flight path, a
veto paddle was installed immediately upstream of the TOF detector.

Custom digitizer software was used to run the 
digitizer in two complementary modes, referred to as "DPP mode" and "Waveform 
mode". In DPP mode, triggers were initiated by the digitizer's onboard
peak-sensing firmware. For each trigger, several quantities were recorded: the trigger 
timestamp, two charge integrals over the detected peak with different
integration ranges (32 ns for the short integral, 100 ns for the long integral),
and a 96-ns portion of the raw digitized waveform, referred to as a ``wavelet".
DPP mode was used for the vast majority of the 
experiment and accounts for $\approx$99\% of the total data volume. In waveform mode, 
the digitizer performs no peak-sensing and was externally triggered. Upon 
triggering, the trigger timestamp and a very long wavelet (60 $\upmu$s) 
were recorded. While waveform mode data accounts for only $\approx$1\% of the total data, 
the instantaneous data rate is much higher than in DPP 
mode because hundreds of $\upmu$s of consecutive waveform samples are 
stored. Roughly once every three seconds, the digitizer was switched to 
waveform mode for one macropulse, then switched back to DPP mode as quickly as
possible (10-40 ms, depending on run configuration).  

Except for the oxygen and rhodium samples, all samples were prepared as right
cylinders 8.25 mm in diameter and ranging from 10-27 mm in length (see
Table \ref{SampleTable} for sample characteristics and Fig. \ref{SamplesImage}
for sample images). A natural-abundance sample
was also prepared for each element under study as were two natural carbon
samples and a natural lead sample for benchmarking to literature. The samples
were inserted into styrofoam sleeves and seated in the cradles of the sample
changer. This design minimizes the amount of non-target mass proximate to the
neutron beam path.

\begin{table*}[ht]
    \caption{Sample Characteristics}
    \label{SampleTable}
    \begin{center}
        \begin{tabular}{ c c c c c c c }
            \hline
            Isotope & Nat. Abund. [\%] & Sample Purity [\%] & Length [mm] & Diameter
            [mm] & Mass [g] & $\rho_{areal}$ [$\frac{mol}{cm^{2}}]$\\
            \hline

            $^{\text{nat}}$C & - & - & 13.66$\pm$0.02 & 8.26$\pm$0.005 & 1.2363
            & 0.19210 \\
            $^{\text{nat}}$C & - & - & 27.29$\pm$0.02 & 8.26$\pm$0.005 & 2.4680
            & 0.38345\\

            H$_{2}$$^{\text{nat}}$O & - & - & 20.0 & 10.0 & N/A & [insert]\\
            D$_{2}$$^{\text{nat}}$O & - & - & 20.0 & 10.0 & N/A & [insert]\\
            H$_{2}$$^{18}$O & 0.20 & 99 & 20.0 & 10.0 & N/A & [insert]\\

            $^{58}$Ni & 68.077 & 99.6 & 7.97$\pm$0.03 & 8.18$\pm$0.02 &
            3.6438 & 0.11965\\
            $^{\text{nat}}$Ni & - & - & 8.00$\pm$0.03 & 8.20$\pm$0.02 &
            3.6898 & 0.11917\\
            $^{64}$Ni & 0.926 & 92.2 & 7.96$\pm$0.02 & 8.20$\pm$0.04 &
            3.9942 & 0.11918\\

            $^{103}$Rh & 100 & 99.9 & 2.03$\pm$0.01 & 10.20$\pm$0.02 & 2.8359 & 0.02426\\

            $^{112}$Sn & 0.97 & 99.9 & 13.65$\pm$0.03 & 8.245$\pm$0.005 &
            4.9720 & 0.083321\\
            $^{\text{nat}}$Sn & - & - & 13.68$\pm$0.03 & 8.245$\pm$0.005 &
            5.3263 & 0.084145\\
            $^{124}$Sn & 5.79 & 99.9 & 13.73$\pm$0.03 & 8.245$\pm$0.005 &
            5.5492 & 0.083988\\

            $^{\text{nat}}$Pb & - & - & 10.07$\pm$0.02 & 8.27$\pm$0.01 & 6.130 &
            0.055077\\

            \hline
        \end{tabular}
    \end{center}
\end{table*}

The oxygen isotopes were prepared as water samples to increase the areal density
of atoms and for ease of handling. Each water sample was contained by a
cylindrical brass vessel with very thin (0.002 inch) brass endcaps, minimizing
beam attenuation in the vessel. Oxygen cross sections were calculated by
subtracting the well-known hydrogen cross section from the raw H$_{2}$O result
(we used H \tots data sets from Clement \cite{Clement1972} and Abfalterer 
\cite{Abfalterer2001}, which together cover the range $0.5 \leq E_n \leq 500$ MeV
and are in excellent agreement where their energy ranges overlap). Because of
the additional uncertainty inherent in this kind of subtractive \tots
determination, a D$_{2}^{\text{nat}}$O sample from which the literature \tots for
D$_{2}$ could be subtracted was prepared as an additional cross-check. Due to its poor 
machining properties, the rhodium
sample was prepared by stacking a series of thin rhodium discs rather than by
producing a fused cylinder. These discs were contained by a cylindrical plastic
case with open ends, similar to the styrofoam sleeves.

The sample configuration for each run varied, but generally all six positions on
the sample changer were used. For the solid targets, a typical configuration was
to place an empty styrofoam sample sleeve in the first sample-changer cradle as
the ``blank", the $^{\text{nat}}$C and $^{\text{nat}}$Pb samples in the second and third
cradles, and the samples of interest (e.g., $^{58}$Ni, $^{\text{nat}}$Ni, $^{64}$Ni) in
the fourth, fifth, and sixth cradles. For water samples, an empty brass vessel
was placed in the first cradle to serve as the blank.

\begin{figure}
    \includegraphics[scale=0.23]{figures/AllIsotopicSamples.jpg}
    \caption{(Color online) Section (a): the ${^{112,nat,124}}$Sn samples. Section (b): the 
        vessels used to hold water samples for the ${^{nat, 18}}$O \tots measurement. 
        Section (c): ${^{58,nat,64}}$Ni samples, shown end-on.}
    \label{SamplesImage}
\end{figure}

\section{Analysis}

The fundamental quantity of interest, \tot, is related to the flux
loss through a sample by:
\begin{equation}
I_{t} = I_{0}e^{-{\ell\rho\sigma_{tot}}}
\end{equation}
or, equivalently,
\begin{equation}
    \tot = -\frac{1}{\ell\rho}ln\left(\frac{I_{t}}{I_{0}}\right)
\end{equation}
where $I_{0}$ is the neutron flux entering the sample, $I_{t}$ is the neutron
flux transmitted through the sample without interaction, $\rho$ is the number
density of nuclei in the sample, and $\ell$ is the sample length. For thin
or low-density samples, flux attenuation through the sample will be small
(e.g., $\approx$13\% for our Ni samples at 100 MeV) and a large number
of counts will be required to determine the cross section to high
precision.

To identify valid neutron events and precisely determine the TOF start (micropulse start 
time) and TOF stop time (arrival at TOF detector) for each event, a series of corrections 
are required.  First, each event was assigned to the correct macropulse.
Time offsets (accounting for cable and
electronics delay) were applied, coarsely synchronizing all detectors with
facility-provided pulses indicating the arrival time of proton micropulses on the
tungsten target, so-called ``$T_{0}$" pulses.
To improve the detector- and digitizer-associated timing resolution for each TOF
event, the digitized waveform for each event was passed 
through an offline software CFD algorithm, improving the timing resolution 
for TOF-detector events to 0.52 ns FWHM (see Figs. \ref{LRTimeDifferenceLinear}
and \ref{DifferenceThresholdsFit}).

\begin{figure*}
    \includegraphics[scale=0.3]{figures/Difference_Linear.png}
    \caption{(Color online) For all events in a diagnostic data run, the timing difference   
        between the left and right PMTs of the TOF detector is shown.
        A gaussian fit (in red) to these events reveals a 93-ps delay of the right PMT with 
    respect to the left PMT and a left-right time difference FWHM of 0.52 ns.}
    \label{LRTimeDifferenceLinear}
\end{figure*}

\begin{figure*}
    \includegraphics[scale=0.3]{figures/DifferenceThresholdsFit.png}
    \caption{(Color online) As a follow-up to Fig. \ref{LRTimeDifferenceLinear},
        all events from a diagnostic run were segregated by neutron energy and the
        time difference between the left and right PMTs of the TOF detector
        was calculated. The FWHM of these time differences are plotted (black
        points) and fitted with a hyperbolic curve (red line). For low-energy
        events, the time resolution is poorer due to the lower signal amplitude and
        a larger traversal time for neutrons through the 1 inch detector. As energy
        increases, the FWHM time resolution approaches its asymptotic minimum at 0.34
    ns (grey dashed line).}
    \label{DifferenceThresholdsFit}
\end{figure*}

To improve the micropulse time resolution and test the stability of the
micropulse period during a macropulse, all $\gamma$ rays for a given macropulse
were identified and their average TOF calculated. The difference between this
time and the expected TOF (given the TOF detector distance and speed of light)
was applied to all events in that macropulse (see Fig.
\ref{TimingCorrectionStudy}). After these corrections, the total TOF resolution
(the FWHM of the $\gamma$-ray peak in the TOF spectra) ranged from
0.60-0.85 ns over the series of \tots measurements and is comparable to the resolution from 
our digitizer-mediated Ca experiment from 2008 \cite{Shane2010}.
To contextualize, for a 100-MeV neutron and a TOF detector distance of 27 meters, a TOF 
uncertainty of 0.80 ns translates to an energy resolution of $\approx$900 keV.
For neutrons below $\approx$20 MeV, the TOF time resolution worsens as the traversal time 
through the 1-inch thickness of the TOF detector becomes non-negligible.
However, because the TOF of these neutrons is already several hundred ns or
longer, the relative energy resolution ($\frac{\Delta E}{E}$) is
superior at low energies: for a 5 MeV neutron with a 0.82 ns detector-traversal time and
an inherent TOF resolution of 0.80 ns, $\Delta E$ is 13 keV. These energy uncertainties
have been properly propagated through subsequent analysis into our \tots results below.

Precise determination of the TOF distance was done by comparing our measured \totEs data
for $^{\text{nat}}$C with that from literature from 3-15 MeV, the results of which are
shown in Fig. \ref{DistanceStudy}. The distance was determined as [insert
updated value]$\pm$1 cm for the Ni and Rh measurement and [insert updated value]$\pm$1 cm 
for the Sn and O measurement.

Because events are not processed instantaneously, there is a brief period,
referred to as the "analytic deadtime", after each event trigger during which
the digitizer is busy processing that trigger. Any newly-arriving events in this
period will be ignored, privileging events arriving earlier and thus distorting
the cross sections. To remedy this, a deadtime correction for
each target must be calculated and applied according to standard techniques
\cite{Moore1980}. Assuming negligible variation in beam flux between micropulses
(an assumption investigated below), the fraction of time $F[i]$ that the digitizer is dead 
for a given time bin $i$ can be calculated:
\begin{equation}
    F_{i} = \sum^{N-1}_{j=0} R_{(i-j)\text{ mod N}}\times P_{j}
\end{equation}
where $N$ is the number of time bins in the micropulse, $R_{x}$ is the rate of
detected events per micropulse in bin $x$, and $P_{j}$ is the probability that the
digitizer is still busy from a trigger $j$ bins ago. To model $P_{j}$, we
employed a logistic function and fitted it to the observed spectrum for time
differences between consecutive events (see Fig.
\ref{TimeDifferenceBetweenEvents}). For a given bin $i$, the fraction of time that the 
digitizer is dead, $F_{i}$, is in essence a discrete convolution of the
\textit{measured} TOF spectrum with $P_{j}$. Note that except for the first and
last micropulses in a macropulse, micropulses are consecutive and thus deadtime effects can
``wrap around" from the end of one micropulse to the next. For these wrap-around
contributions (that is, $j>i$), the (mod N) term ensures that the bin referred
to by $i-j$ is non-negative and has physical meaning as a time bin from the previous 
micropulse.

\begin{figure}
    \includegraphics[scale=0.24]{figures/TimeDifferenceBetweenEvents.png}
    \caption{(Color online) The time difference between adjacent TOF-detector
    events for a single run is plotted (black histogram). Below a certain
minimum time difference (the ``deadtime"), no events are recorded. A logistic
fit (red) models the detector's deadtime response and is used to generate a
deadtime correction. The underlying linearly-decreasing count rate (gray dashed
line) in incorporated into the logistic model. From the fit, a mean deadtime of
228.1 ns was extracted for the Sn and O run configurations (a similar
procedure was used to recover a deadtime of [insert deadtime] for the Ni and Rh
run configurations).}
    \label{TimeDifferenceBetweenEvents}
\end{figure}

\begin{figure*}
    \includegraphics[scale=0.3]{figures/distanceStudy.png}
    \caption{(Color online) Results of a study to determine the distance between
    the neutron source and the TOF detector for the Ni/Rh running configuration
are shown. First, a plausible range of flight path distances was selected based
on rough estimation during the experiment (26.97-27.21 m). For each of several
distances in this range, the \totEs for natural carbon was calculated in the
resonance region (3-15 MeV). then the RMS difference between our cross section
(calculated using that distance) and literature data from Abfalterer
\cite{Abfalterer2000, Abfalterer2001} was calculated (shown as black points in
the figure). A quartic fit to these RMS data is shown in red. By minimizing the
RMS difference, we calculate a flight path distance of 2709$\pm$1 cm.}
    \label{DistanceStudy}
\end{figure*}

Typical values for $F_{i}$ in our experiment are shown in Fig.
\ref{ExampleDeadtimeSpectrum}. Because trigger processing is done quickly in
firmware onboard the digitizer, the per-event deadtimes affecting our
measurement ranged from 150-230 ns, much smaller than the several-$\upmu$s
deadtime-equivalents in previous analog experiments \cite{Finlay1993,
Abfalterer2001}.

Once the fraction dead was identified for each time bin, the total number of
events detected in that bin, $N_{m}[i]$, was corrected to the \textit{true}
number of events in that bin $N_{t}[i]$ that would have been detected absent an
analytic deadtime:
\begin{equation}
    N_{t}[i] = -ln\left(\frac{1-\frac{N_m[i]}{M}}{(1-F_{i})\times M}\right)
\end{equation}
where M is the total number of micropulse periods. The difference between
uncorrected and analytic-deadtime-corrected TOF spectra is shown in Fig.
\ref{CorrectionEffectOnTOF}. At large TOFs (low energies) the correction is as low as a
few percent, but at small TOFs (high energies) when the digitizer is still dead
from the gamma flash and high-energy neutrons, the correction is significant
($\approx$20\% for our Ni/Rh runs, and $\approx$40\% for our Sn/O runs). These 
corrections are themselves far lower than the typical
analytic deadtime correction required when using analog approaches \cite{Finlay1993,
Abfalterer2001} that were as high as 500\%.% The effect of this correction on the final 
%cross section results is shown in Fig. \ref{DeadtimeEffectOnCS}.

[insert information about constant rate and discarding first 40 micropulses]

\begin{figure*}
    \includegraphics[scale=0.3]{figures/exampleDeadtimeSpectrum.png}
    \caption{(Color online) Using TOF data from a typical run, the probability that a given 
        TOF bin is ``dead" is shown for the blank sample (in red) and the $^{\text{nat}}$C   
        sample (in blue). Note the sharp rise at 90 ns corresponding to the
        $\gamma$-ray flash, the gradual increase from 90-245 ns corresponding to
        the arrival of high energy neutrons, and the sharp fall at 245 ns
        corresponding to the elapse of the $\gamma$-ray ``shadow". Only high-
        energy neutrons have a probability-dead $>$10\% (cf. the 50-80\%
        probability-dead of \cite{Finlay1993, Abfalterer2001}).
    }
    \label{ExampleDeadtimeSpectrum}
\end{figure*}

\begin{figure*}
    \includegraphics[scale=0.3]{figures/CorrectionEffectOnTOF.png}
    \caption{(Color online) A typical TOF spectrum from the Ni/Rh
        run configuration is shown, before (in blue) and after (in red) analytic
        deadtime correction. Relevant neutron energies are indicated above the spectra.
        For this digitizer configuration, the mean deadtime was 155 ns (see Fig.
        \ref{TimeDifferenceBetweenEvents} for details on mean deadtime determination).
        Note that at 245 ns, there is an
        obvious defect in the uncorrected spectrum is repaired in the corrected
        spectrum. The defect
        corresponds to the elapse of the deadtime ``shadow" from the $\gamma$
        ray flash that came 155 ns earlier at $\approx$90 ns (not pictured).
    }
    \label{CorrectionEffectOnTOF}
\end{figure*}

[edits needed]
In addition to analytic deadtime, an additional deadtime associated with digitizer readout 
to the data acquisition computer (DAQ), was identified. In our Each pair of
digitizer channels shares a common buffer for storing events; upon reaching a
user-defined threshold, buffer contents are read out to the DAQ as packets.
During readout, the digitizer is ...something.
This is likely the largest source of systematic error in our measurement.

During analysis, it was noted that occasionally (\~1 in 400 macropulses), one or two 
adjacent macropulses would have an abnormally small number of flux monitor events or 
TOF events. The frequency of these "data dropouts" was similar to the rate of
switching between DPP and waveform modes and we suspect it is related to edge
case behavior right before or after a mode switch. To mitigate this issue,
any macropulse that had less than 50\% of the average event rate in either the
flux monitor or TOF detector channel was ignored during cross section calculation.
[edits needed]

After applying these corrections, the veto and integrated charge gates are applied to 
all events and surviving events are populated into TOF spectra (see Fig.
\ref{ExampleTOFSpectrum}). Next, room background (responsible for 0.1\% to 1\% 
of event rate, depending on energy) is subtracted and spectra are mapped to the
energy domain.

\begin{figure*}
    \includegraphics[scale=0.3]{figures/exampleTOFSpectrum.png}
    \caption{(Color online) TOF spectra after analytic deadtime correction and
        veto and integrated charge gating for the blank sample (in
        red) and for the $^{\text{nat}}$C sample (in blue), from the Ni/Rh experiment.
        The $\gamma$-ray peak is visible as a sharp spike at 90 ns, followed by
        the highest-energy neutrons at $\approx$130ns. Small spikes spaced 60 ns
        apart occur throughout the spectrum (visible before 90 ns and after 1500
        ns). These are $\gamma$ rays caused by a continuous ``drip" of protons onto the 
        tungsten target due to mistuning of the proton buncher; their
        effect on the calculated cross sections is negligible.
    }
    \label{ExampleTOFSpectrum}
\end{figure*}

From these energy spectra, the cross section was calculated, bin-wise, as follows:
$$
\tot = -\frac{1}{\ell\rho_{n}}
\ln \left(\frac{I_{0}}{I_{s}}\times\frac{M_{s}}{M_{0}}\right)
$$
where $\frac{I_{0}}{I_{s}}$ is the ratio of counts in the energy spectra between 
the blank and sample, $\frac{M_{s}}{M_{0}}$ is the ratio of counts in the
monitor detector between the sample and blank (for flux normalization), $\ell$ is the length 
of the sample, and $\rho_{n}$ is the number density of atoms in the sample.

\section{Experimental Results}

To verify our approach, we first benchmarked our \totEs measurements of natural samples
($^{\text{nat}}$C, $^{\text{nat}}$Ni, $^{\text{nat}}$Sn, and
$^{\text{nat}}$Pb) against the high-precision data sets on natural samples from Finlay
\cite{Finlay1993} and Abfalterer \cite{Abfalterer2001}, shown in figures
\ref{LiteratureBenchmarking}. As an additional diagnostic, we compared both our natural 
carbon targets against each other (see Fig. \ref{CarbonBenchmarking}).

\begin{figure*}
    \includegraphics[scale=0.35]{figures/LiteratureBenchmarking.png}
    \caption{(Color online) }
    \label{LiteratureBenchmarking}
\end{figure*}

\begin{figure*}
    \includegraphics[scale=0.30]{figures/relativeDiff_longCarbonShortCarbon.png}
    \caption{(Color online) Relative difference of cross sections (red line) of
        our two lengths of $^{\text{nat}}$C, (13.7 mm and 27.3 mm), shown from 3-400
        MeV. Total error
        (including statistical and systematic) is indicated by the blue
        dotted region. Systematic error only (due to uncertainty in the areal
        density) is shown by the red hatched region (very small and immediately adjacent to 
        the red line). Clearly, statistical error dominates for this relative
        difference.
        Despite the low statistics of this diagnostic run (only 1.5 hours
        beam-on-target for each sample), the high efficiency of the
        digitizer-enabled approach means that the relative difference can be resolved to 
        $\approx$1\% throughout the entire energy range.
    }
    \label{CarbonBenchmarking}
\end{figure*}


\begin{figure*}
    \includegraphics[scale=0.35]{figures/FourPanelO.png}
    \caption{(Color online) Neutron total cross sections for $^{16,18}$O.
     Panel one shows our measurement (in red) and literature data from
     Abfalterer (in
     black), where the $^{18}$O values have been shifted up by 1 barn for
     readability. Panel two shows the relative difference between our 
     measurement and the literature data from the first panel. Panels three and
     four show the corresponding absolute and relative cross sections after a
     Pb-C correction, derived from literature data and our Sn dataset,
     are applied.
    }
\end{figure*}

\begin{figure*}
    \includegraphics[scale=0.35]{figures/FourPanelNi.png}
    \caption{(Color online) Neutron total cross sections for $^{58,64}$Ni.
     Panel one shows our measurement (in red) and literature data from
     Abfalterer (in % black).
     Panel two shows the relative difference between our 
     measurement and the literature data from the first panel. Panels three and
     four show the corresponding absolute and relative cross sections after a
     Pb-C correction, derived from literature data and detailed in the text,
     are applied.
    }
\end{figure*}

\begin{figure*}
    \includegraphics[scale=0.35]{figures/FourPanelSn.png}
    \caption{(Color online) Neutron total cross sections for $^{112,124}$Sn.
     Panel one shows our measurement (in red) and literature data from
     Finlay (in % black).
     Panel two shows the relative difference between our 
     measurement and the literature data from the first panel. Panels three and
     four show the corresponding absolute and relative cross sections after a
     Pb-C correction, derived from literature data and detailed in the text,
     are applied.
    }
\end{figure*}

\begin{figure*}
    \includegraphics[scale=0.35]{figures/FourPanelNiIsotopes.png}
    \caption{(Color online) Neutron total cross sections for $^{58,64}$Ni.
     Panel one shows our measurement (in red) and literature data from
     Abfalterer (in % black).
     Panel two shows the relative difference between our 
     measurement and the literature data from the first panel. Panels three and
     four show the corresponding absolute and relative cross sections after a
     Pb-C correction, derived from literature data and detailed in the text,
     are applied.
    }
\end{figure*}

\begin{figure*}
    \includegraphics[scale=0.35]{figures/FourPanelSnIsotopes.png}
    \caption{(Color online) Neutron total cross sections for $^{112,124}$Sn.
     Panel one shows our measurement (in red) and literature data from
     Finlay (in % black).
     Panel two shows the relative difference between our 
     measurement and the literature data from the first panel. Panels three and
     four show the corresponding absolute and relative cross sections after a
     Pb-C correction, derived from literature data and detailed in the text,
     are applied.
    }
\end{figure*}





Cross sections for all targets are shown in Fig. \ref{FourPanelO, FourPanelNi,
FourPanelRh, FourPanelSn}.

\section{Discussion}

[Discussion]

\section{DOM Analysis}

[Preliminary DOM Analysis for oxygen]

\section{Conclusion}

[Conclusion]

\section{Acknowledgements}

\bibliography{references}
\begin{thebibliography}{32} \expandafter\ifx\csname
        natexlab\endcsname\relax\def\natexlab#1{#1}\fi \expandafter\ifx\csname
        bibnamefont\endcsname\relax \def\bibnamefont#1{#1}\fi
        \expandafter\ifx\csname bibfnamefont\endcsname\relax
        \def\bibfnamefont#1{#1}\fi \expandafter\ifx\csname
        citenamefont\endcsname\relax \def\citenamefont#1{#1}\fi
        \expandafter\ifx\csname url\endcsname\relax \def\url#1{\texttt{#1}}\fi
        \expandafter\ifx\csname urlprefix\endcsname\relax\def\urlprefix{URL
        }\fi \providecommand{\bibinfo}[2]{#2}
        \providecommand{\eprint}[2][]{\url{#2}}

\end{thebibliography}

\end{document}
